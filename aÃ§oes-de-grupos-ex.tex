%---------------------------------------------------------------------------------------------------------------------!Draft!-----------------------------------------------------------------------------------------------------------------
\subsection{Ações de grupos}
\label{ações-de-grupo-ex}
%\begin{titlemize}{Lista de dependências}
	%\item \hyperref[dependecia1]{Dependência 1};\\ %'dependencia1' é o label onde o conceito Dependência 1 aparece (--à arrumar um padrão para referencias e labels--) 
	%\item \hyperref[]{};\\
% quantas dependências forem necessárias.
%\end{titlemize}

\begin{ex}[Ações de grupos]
	\begin{itemize}
    
	    \item[1.] $G\circlearrowright X \\
                g\cdot x = x$\\
                $\forall \  G$ grupo e $X$ conjunto, é a ação trivial que induz a permutação identidade em $X$

        \item[2.] $G\circlearrowright G\\
                g\cdot h = gh$\\
                $\forall g,h \in G$ é a ação de multiplicação à esquerda
        
	  \item[3.]  $G\circlearrowright G\\
                h\cdot g = hgh^{-1}$\\
                $\forall g,h \in G$ é a ação da conjugação de $g$ por $h$

        \item[4.] $S_n \circlearrowright \{1.\dots,n\}$\\
                $g\cdot x = g(x)$\\
                O grupo simétrico (e seus subgrupos) age no conjunto $\{1.\dots,n\}$ permutando seus elmenetos.
        
        \item[5.] $\mathbb{R}^n\circlearrowright \mathbb{R}^n\\
                x\cdot y = x+y$, $\forall x,y\in \mathbb{R}^n$\\
                É a ação de translação em $\mathbb{R}^n$

        \item[6.] $S_n \circlearrowright \mathbb{R}[x_1,\dots,x_n]$\\
                $\sigma\cdot f(x_1,\dots,x_n) = f(x_{\sigma(1)},\dots,x_{\sigma(n)})$\\
                O grupo simétrico age no anel de polinômios com coeficientes em $\mathbb{R}$ permutando as variáveis.


    \end{itemize}
\end{ex}
 

\begin{titlemize}{Lista de consequências}
	\item \hyperref[ações-de-grupos-propriamente-descontínuas-def]{Ações de grupos propriamente descontínuas};
\end{titlemize}