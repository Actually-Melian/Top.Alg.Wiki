%---------------------------------------------------------------------------------------------------------------------!Draft!-----------------------------------------------------------------------------------------------------------------
\subsection{Ações de grupo}
\label{ações-de-grupo-def}

%\begin{titlemize}{Lista de dependências}
	%\item \hyperref[dependecia1]{Dependência 1};\\ %'dependencia1' é o label onde o conceito Dependência 1 aparece (--à arrumar um padrão para referencias e labels--) 
	%\item \hyperref[]{};\\
% quantas dependências forem necessárias.
%\end{titlemize}

\begin{defi}[Ação de Grupo]
	Seja $G$ um grupo com elemto neutro 1 e $X$ um conjunto. Uma ação do grupo $G$ sobre o conjunto $X$ pela esquerda é uma função\\
 $\psi: G\times X \longrightarrow X$ satisfazendo:
 
 \begin{itemize}
     \item[(i)] $\psi(1,x) = x \  \forall\  x \in X$
     \item[(ii)] $\psi(g,\psi(h,x)) = \psi(gh,x) \ \forall\ x \in X,\ \forall\ g,h \in G$
 \end{itemize}
 Neste caso, dizemos que $G$ age pela esquerda em $X$.\\
 

\noindent Analogamente,\\
 Uma ação do grupo $G$ sobre o conjunto $X$ pela direita é uma função\\
 $\psi: X\times G \longrightarrow X$ satisfazendo:
 
 \begin{itemize}
     \item[(i)] $\psi(x,1) = x ,\  \forall\  x \in X$
     \item[(ii)] $\psi(\psi(g,x), h) = \psi(x,gh), \ \forall\ x \in X,\ \forall\ g,h \in G$
 \end{itemize}
 Neste caso, dizemos que $G$ age pela direita em $X$.

\end{defi}

Outras notações: $\psi(g,x) = \psi_{g}(x)$ ou simplesmente $g\cdot x$ se a ação (à esquerda) estiver explícita no contexto.\\
Além disso, escrevemos $G\circlearrowright X$ para denotar que $G$ age em $X$.


\begin{titlemize}{Lista de consequências}
	\item \hyperref[ações-de-grupos-propriamente-descontínuas-def]{Definição de ação de grupo propriamente descontínua};\\ %'consequencia1' é o label onde o conceito Consequência 1 aparece
\end{titlemize}

%[Bianca]: é mais fácil criar a lista de dependências do que a de consequências.