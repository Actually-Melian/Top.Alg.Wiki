%---------------------------------------------------------------------------------------------------------------------!Draft!-----------------------------------------------------------------------------------------------------------------
\subsection{Ações de Grupos Propriamente descontínuas}
\label{ações-de-grupo-propriamente-descontínuas-def}
\begin{titlemize}{Lista de dependências}
	\item \hyperref[ações-de-grupo-def]{Ações de grupos};\\ %'dependencia1' é o label onde o conceito Dependência 1 aparece (--à arrumar um padrão para referencias e labels--) 
% quantas dependências forem necessárias.
\end{titlemize}
\begin{defi}[Ação propriamente descontínua]
	Dizemos que a ação:
	\begin{alignat*}{1}
    		\psi_{g}: G\times X &\longrightarrow X&\\
    		\psi_{g}(x) ^&\longmapsto gx
	\end{alignat*}

é própriamente descontínua se $\forall x \in X$, existe um aberto $U \subset X$ de $x$ tal que:
\[gU \cap U \neq \emptyset \  \Rightarrow g = 1\]

\noindent onde $gU = \{gy\  |\  y \in U\} = \psi_{g}(U) \subset U$ e
\begin{alignat*}{1}
    \psi_{g}: X&\longrightarrow X&\\
    \psi_{g}(y)^&\longmapsto gy
\end{alignat*}
\end{defi}
\\
Uma possível intuição geométrica para uma ação propriamente descontínua, é interpretá-la como uma função $\psi$ que leva os pontos $x$ do espaço para "longe" de sua posição inicial.\\

\begin{titlemize}{Lista de consequências}
	\item \hyperref[ações-de-grupos-e-recobrimentos-prop]{Ações de grupos propriamente descontínuas geram recobrimento};\\ %'consequencia1' é o label onde o conceito Consequência 1 aparece
    \item \hyperref[ações-de-grupos-e-gr-fundamental-prop]{Ações de grupo e grupos fundamentais}
\end{titlemize}

%[Bianca]: é mais fácil criar a lista de dependências do que a de consequências.
