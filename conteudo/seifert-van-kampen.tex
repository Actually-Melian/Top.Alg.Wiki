\section{Teorema de Seifert-Van Kampen}
\label{teorema-de-seifert-van-kampen}

\begin{titlemize}{Lista de Dependências}
	\item \hyperref[grupo-fundamental]{Grupo Fundamental};\\
	\item \hyperref[espaco-de-recobrimento]{Espaço de recobrimento};\\
    \item \hyperref[ações-de-grupos-e-recobrimentos]{Ações de grupos e recobrimentos};
\end{titlemize}

O teorema de Seifert-Van Kampen é um importante resultado da topologia algébrica que nos permite calcular o grupo fundamental de diversos espaços topológicos razoáveis a partir dos grupos fundamentais de certos subespaços.

A menos que explicitamente mencionado, no que segue assumiremos que o espaço base $X$ é razoável, isto é, $X$ é conexo, localmente conexo por caminhos, e localmente semi-simplesmente conexo. Como visto em seções anteriores, $X$ conexo e localmente conexo por caminhos nos garante a exisência e unicidade de levantamentos, $X$ localmente semi-simplesmente conexo nos garante a exisência de um recobrimento 1-conexo.

\input{}