\section{Teorema de Seifert-Van Kampen}
\label{teorema-de-seifert-van-kampen}

\begin{titlemize}{Lista de Dependências}
	\item \hyperref[grupo-fundamental]{Grupo Fundamental};\\
	\item \hyperref[espaco-de-recobrimento]{Espaço de recobrimento};\\
    	\item \hyperref[ações-de-grupos-e-recobrimentos]{Ações de grupos e recobrimentos};\\
    	\item \hyperref[recobrimento-universal]{Recobrimento Universal};
\end{titlemize}

O teorema de Seifert-Van Kampen é um importante resultado da topologia algébrica que nos permite calcular o grupo fundamental de diversos espaços topológicos razoáveis a partir dos grupos fundamentais de certos subespaços.

A menos que explicitamente mencionado, no que segue assumiremos que o espaço base $X$ é razoável, isto é, $X$ é conexo, localmente conexo por caminhos, e localmente semi-simplesmente conexo. Como visto em seções anteriores, $X$ conexo e localmente conexo por caminhos nos garante a exisência e unicidade de levantamentos, $X$ localmente semi-simplesmente conexo nos garante a existência do recobrimento universal, que será denotado por $p:(\tilde X, \tilde x_0) \longrightarrow (X, x_0)$.
\subsection{Automorfismo de um recobrimento}
\label{automorfismo-de-recobrimento-def}
\begin{titlemize}{Lista de dependências}
	\item \hyperref[ações-de-grupo-def]{Ações de grupo};
\end{titlemize}
\begin{defi}[Automorfismo de um recobrimento]
    Um automorfismo de um recobrimento $q:E \longrightarrow X$ é um homeomorfismo $\phi:E \longrightarrow E$ tal que:
    \[\begin{tikzcd}
	E && E \\
	\\
	& X
	\arrow["\phi", from=1-1, to=1-3]
	\arrow["q"', from=1-1, to=3-2]
	\arrow["q", from=1-3, to=3-2]
    \end{tikzcd}\]
\end{defi}

Denotamos o grupo formado por todos os automorfismos de um recobrimento, também chamado de grupo de transformações de Deck, por $Aut(E, q, X)$. Por definição, para todo $x \in X$, a ação $D \circlearrowright q^{-1}(x)$ é dada por $$\phi \cdot e = \phi(e)$$

\begin{titlemize}{Lista de consequências}
	\item \hyperref[acao-de-automorfismos-e-livre-prop]{A ação do grupo de automorfismos é livre};\\
    \item \hyperref[acao-de-automorfismo-transitiva-prop]{Quando a ação do grupo de automorfismos é transitiva sobre as fibras?};\\
    \item \hyperref[g-recobrimento-regular-def]{G-recobrimento regular};
\end{titlemize}
%---------------------------------------------------------------------------------------------------------------------!Draft!-----------------------------------------------------------------------------------------------------------------
\subsection{G-recobrimento regular}
\label{g-recobrimento-regular-def}
\begin{titlemize}{Lista de dependências}
	\item \hyperref[automorfismo-de-recobrimento-def]{Automorfismo de um recobrimento};
\end{titlemize}
\begin{defi}[G-recobrimento regular]
	Um G-recobrimento regular, ou recobrimento G-regular ou ainda recobrimento G-normal de X é um recobrimento $q:E \longrightarrow X$ tal que $Aut(E, q, X) = G$, e a ação $G \circlearrowright q^{-1}(x)$ é livre e transitiva $\forall x \in X$.
\end{defi}

\begin{titlemize}{Lista de consequências}
	\item \hyperref[g-recobrimentos-e-epimorfismos-prop]{G-recobrimentos 0-conexos e epimorfismos do grupo fundamental em G};
\end{titlemize}

