\section{Teorema de Seifert-Van Kampen}
\label{teorema-de-seifert-van-kampen}

\begin{titlemize}{Lista de Dependências}
	\item \hyperref[grupo-fundamental]{Grupo Fundamental};\\
	\item \hyperref[espaco-de-recobrimento]{Espaço de recobrimento};\\
    	\item \hyperref[ações-de-grupos-e-recobrimentos]{Ações de grupos e recobrimentos};\\
    	\item \hyperref[recobrimento-universal]{Recobrimento Universal};
\end{titlemize}

O teorema de Seifert-Van Kampen é um importante resultado da topologia algébrica que nos permite calcular o grupo fundamental de diversos espaços topológicos razoáveis a partir dos grupos fundamentais de certos subespaços.

A menos que explicitamente mencionado, no que segue assumiremos que o espaço base $X$ é razoável, isto é, $X$ é conexo, localmente conexo por caminhos, e localmente semi-simplesmente conexo. Como visto em seções anteriores, $X$ conexo e localmente conexo por caminhos nos garante a exisência e unicidade de levantamentos, $X$ localmente semi-simplesmente conexo nos garante a existência do recobrimento universal, que será denotado por $p:(\tilde X, \tilde x_0) \longrightarrow (X, x_0)$.
\subsection{Automorfismo de um recobrimento}
\label{automorfismo-de-recobrimento-def}
\begin{titlemize}{Lista de dependências}
	\item \hyperref[ações-de-grupo-def]{Ações de grupo};
\end{titlemize}
\begin{defi}[Automorfismo de um recobrimento]
    Um automorfismo de um recobrimento $q:E \longrightarrow X$ é um homeomorfismo $\phi:E \longrightarrow E$ tal que:
    \[\begin{tikzcd}
	E && E \\
	\\
	& X
	\arrow["\phi", from=1-1, to=1-3]
	\arrow["q"', from=1-1, to=3-2]
	\arrow["q", from=1-3, to=3-2]
    \end{tikzcd}\]
\end{defi}

Denotamos o grupo formado por todos os automorfismos de um recobrimento, também chamado de grupo de transformações de Deck, por $Aut(E, q, X)$. Por definição, para todo $x \in X$, a ação $D \circlearrowright q^{-1}(x)$ é dada por $$\phi \cdot e = \phi(e)$$

\begin{titlemize}{Lista de consequências}
	\item \hyperref[acao-de-automorfismos-e-livre-prop]{A ação do grupo de automorfismos é livre};\\
    \item \hyperref[acao-de-automorfismo-transitiva-prop]{Quando a ação do grupo de automorfismos é transitiva sobre as fibras?};\\
    \item \hyperref[g-recobrimento-regular-def]{G-recobrimento regular};
\end{titlemize}
\subsection{A açào do grupo de automorfismos é livre}
\label{acao-de-automorfismos-e-livre-prop}
\begin{titlemize}{Lista de dependências}
	\item \hyperref[levantamento-de-funções-prop]{Levantamento de funções};\\
	\item \hyperref[automorfismo-de-recobrimento-def]{Automorfismo de um recobrimento};
\end{titlemize}
\begin{prop}[Nome da Afirmação]
	Dado um recobrimento $q:E \longrightarrow X$ $0$-conexo, então a ação $Aut(E, q, X) \circlearrowright q^{-1}(x)$ é livre para todo $x \in X$. Ou seja, $\forall x \in X$ $\forall e \in q^{-1}(x)$, $\phi \cdot e = e \Longrightarrow \phi = Id$.\\
\end{prop}

\begin{dem}
    Note que $\phi$ é um levantamento de $q$, logo, como $E$ é conexo e $Id:E \longrightarrow E$ também é um levantamento de q, pela unicidade de levantamentos, segue que dados $x \in X$ e $e \in q^{-1}(x)$, então $$\phi \cdot e = e \Longrightarrow \phi(e) = Id(e) \Longrightarrow \phi = Id$$
    % https://q.uiver.app/#q=WzAsNCxbMiwwLCJFIl0sWzIsMiwiWCJdLFswLDIsIkUiXSxbMSw0LCJcXGJ1bGxldCJdLFsyLDAsIklkIiwyXSxbMiwwLCJcXHBoaSIsMCx7Im9mZnNldCI6LTN9XSxbMCwxLCJwIiwyXSxbMiwxLCJwIiwyXV0=
    \[\begin{tikzcd}
    	&& E \\
    	\\
    	E && X \\
    	\arrow["q"', from=1-3, to=3-3]
    	\arrow["Id"', from=3-1, to=1-3]
    	\arrow["\phi", shift left=3, from=3-1, to=1-3]
    	\arrow["q"', from=3-1, to=3-3]
    \end{tikzcd}\]
\end{dem}

\begin{titlemize}{Lista de consequências}
	\item \hyperref[acao-de-automorfismo-transitiva-prop]{Quando a ação do grupo de auitomorfismos é transitiva sobre as fibras?};
\end{titlemize}
\subsection{Quando a ação do grupo de automorfismos é transitiva sobre as fibras?}
\label{acao-de-automorfismo-transitiva-prop}
\begin{titlemize}{Lista de dependências}
    \item \hyperref[levantamento-de-caminhos-prop]{Levantamento de caminhos};\\
	\item \hyperref[automorfismo-de-recobrimento-def]{Automorfismo de um recobrimento};\\
    \item \hyperref[acao-de-automorfismos-e-livre-prop]{A ação do grupo de automorfismos é livre sobre as fibras};
\end{titlemize}
Seja $q:(E, e_0) \longrightarrow (X, x_0)$ um recobrimento 0-conexo pontuado e $D=Aut(E, q, X)$ o grupo de automorfismos desse recobrimento, queremos achar alguma condição necessária e suficiente para que a ação de $D$ seja transitiva sobre as fibras, ou seja, $\forall x \in X$ a ação $D \circlearrowright q^{-1}(x)$ é tal que $\forall e, e' \in q^{-1}(x)$ $\exists \phi \in D$ tal que $\phi(e) = e'$.

Primeiramente, suponha que a ação $D \circlearrowright q^{-1}(x_0)$ é transitiva. Considere a função $\varphi_{e_0}:\pi_1(X, x_0) \longrightarrow D$, onde $\varphi_{e_0}([\alpha]) = d \Longleftrightarrow d \cdot e_0 = \tilde \alpha_{e_0}(1)$.

\begin{af}
    A função $\varphi_{e_0}$ acima está bem definida e é um homomorfismo sobrejetor.
\end{af}

\begin{dem}
    Tome $[\alpha] \in \pi_1(X, x_0)$, como $\alpha(1) = x_0$, segue que $\tilde\alpha_{e_0}(1) \in q^{-1}(x_0)$, temos ainda que $e_0 \in q^{-1}(x_0)$, logo, como a ação $D \circlearrowright q^{-1}(x_0)$ é transitiva por hipótese, seque que $\exists d \in D$ tal que $d \cdot e_0 = \tilde\alpha_{e_0}(1)$.
    
    Vimos ainda que a ação $D \circlearrowright q^{-1}(x_0)$ é livre, donde segue que $d$ é único. Por fim, se $\alpha \sim \alpha'$ $rel$ $\partial I$, segue que $\tilde\alpha_{e_0}(1) = \tilde\alpha_{e_0}'(1)$, logo $\varphi_{e_0}$ não depende da escolha de representantes e está bem definida.

    Tome $d \in D$, como $E$ é $0$-conexo, $\exists \gamma:I \longrightarrow E$ tal que $\gamma(0) = e_0$ e $\gamma(1) = d \cdot e_0$. Seja $\alpha := q \circ \gamma$, então por unicidade de levantamentos segue que $\tilde\alpha_{e_0} = \gamma$, donde $\varphi_{e_0}([\alpha]) = d$ e $\varphi_{e_0}$ é sobrejetora.

    Por fim, tome $[\alpha], [\beta] \in \pi_1(X, x_0)$, então: $$\varphi_{e_0}([\alpha] \cdot [\beta]) = \varphi([\alpha*\beta])$$ e temos que: $$\widetilde{(\alpha * \beta)}_{e_0}(1) = (\tilde\alpha_{e_0}*\tilde\beta_{\tilde\alpha_{e_0}(1)})(1) = \tilde\beta_{\tilde\alpha_{e_0}(1)}(1).$$

    Por outro lado, sejam $$d'e_0 = \tilde\alpha_{e_0}(1)$$ $$d''e_0 = \tilde\beta_{e_0}(1),$$ então veja que: $$\varphi_{e_0}([\alpha]) \cdot \varphi_{e_0}([\beta]) = d'd''$$ e que $$\tilde\beta_{\tilde\alpha_{e_0}(1)} = \tilde\beta_{d'e_0}$$

    Note que: $$q \circ (d'\tilde\beta_{e_0}) = q \circ d' \circ \tilde\beta_{e_0} = q \circ \tilde\beta_{e_0} = \beta,$$ pois $d'$ é um automorfismo de recobrimento. Além disso: $$d'\tilde\beta_{e_0}(0) = d'e_0$$ de modo que, pela unicidade de levantamentos: $$\tilde\beta_{d'e_0} = d'\tilde\beta_{e_0}$$

    Disso segue que: $$\widetilde{(\alpha * \beta)}_{e_0}(1) = \tilde\beta_{d'e_0}(1) = d'\tilde\beta_{e_0}(1) = d'd''e_0$$ e portanto: $$\varphi([\alpha]\cdot[\beta]) = d'd'' = \varphi_{e_0}([\alpha]) \cdot \varphi_{e_0}([\beta]).$$
\end{dem}

\begin{prop}
    O kernel de $\varphi_{e_0}$ é $q_*(\pi_1(E, e_0))$.
\end{prop}

\begin{dem}
Tome $[\alpha] \in Ker(\varphi_{e_0})$, então:\\
    $$\begin{tabular}{l l}
        $[\alpha] \in Ker(\varphi_{e_0})$ & $\Longleftrightarrow                                                     \tilde\alpha_{e_0}(1) = e_0$ \\
                                        & $\Longleftrightarrow \tilde\alpha_{e_0} \in \Omega(E, e_0)$\\
                                        & $\Longleftrightarrow [\tilde\alpha_{e_0}] \in \pi_1(E, e_0)$\\
                                        & $\Longleftrightarrow [\alpha] \in Im(q_*)$
    \end{tabular}$$\\
    de modo que $Ker(\varphi_{e_0}) = q_*(\pi_1(E, e_0))$.
\end{dem}

Note que isso implica que $q_*(\pi_1(E, e_0)) \triangleleft \pi_1(X, x_0)$ e, ainda: $$D \cong \frac{\pi_1(X, x_0)}{q_*(\pi_1(E, e_0))}$$ pelo teorema de isomorfismos.

Reciprocamente, supondo que $q_*(\pi_1(E, e_0)) \triangleleft \pi_1(X, x_0)$, veremos que $D \cong \frac{\pi_1(X, x_0)}{q_*(\pi_1(E, e_0))}$ e age transitivamente em $q^{-1}(x_0)$.

\begin{af}
    Nas condições acima, se $D$ age transitivamente em $q^{-1}(x_0)$, então age transitivamente em todas as fibras.
\end{af}

\begin{dem}
    Como $X$ é conexo por caminhos, segue que $\pi_1(X, x_0) \cong \pi_1(X, x_1)$ $\forall x_1 \in X$, analogamente $\pi_1(E, e_0) \cong \pi_1(E, e_1)$ $\forall e_1 \in E$. Logo, dados $x_1 \in X$ e $e_1 \in E$ temos $\frac{\pi_1(X, x_0)}{q_*(\pi_1(E, e_0))} \cong D \cong \frac{\pi_1(X, x_1)}{q_*(\pi_1(E, e_1))}$ e, como $\frac{\pi_1(X, x_1)}{q_*(\pi_1(E, e_1))}$ age transitivamente em $q^{-1}(x_1)$, da arbitrariedade de $x_1$ segue que $D$ age transitivamente em todas as fibras.
\end{dem}

\begin{lemma}
    Seja $G$ um grupo e suponha que a ação $G \circlearrowright E$ é propriamente descontínua, então $D \cong G$.
\end{lemma}

\begin{dem}
    Seja $\psi:G \longrightarrow D$ a função dada por $\psi(g) = \psi_g$, onde $\psi_g$ é o homeomorfismo determinado pela ação do elemento g em X, $\psi$ é um homomorfismo por se tratar de uma ação.

    Primeiro note que $\psi$ é injetor, afinal, se $\psi_g = Id$, então $g \cdot e = e$, de modo que $g = 1_G$, pois toda ação propriamente descontínua é livre. Além disso, $\psi$ é sobrejetor, afinal dado $\phi \in D$, sabemos que $$q(e_0) = (q \circ \phi)(e_0)$$ pois $\phi$ é automorfismo de recobrimentos. Portanto, $\exists g \in G$ tal que $\phi(e_0) = ge_0$. Logo, como ambos $\psi_g$ e $\phi$ sào levantamentos de $q$, segue da unicidade de levantamentos que $\psi_g = \phi$.
\end{dem}

\begin{af}
    Vale que $(E, e_0) \cong (\frac{\tilde X}{q_*(\pi_1(E, e_0))}, \overline{[c_{x_0}]})$
\end{af}

\begin{af}
    Seja $G = \frac{\pi_1(X, x_0)}{q_*(\pi_1(E, e_0))}$, então a ação $G \circlearrowright E$ dada por $$\overline{[\alpha]} \cdot \overline{[\gamma]} = \overline{[\alpha * \gamma]}$$ é propriamente descontínua com quociente $X$.
\end{af}

\begin{thm}[Condição necessária e suficiente para a ação do grupo de automorfismos ser transitiva sobre as fibras]
	Se $q:E \longrightarrow X$ é um recobrimento $0$-conexo, então a ação $D \circlearrowright q^{-1}(x)$ é transitiva para todo $x \in X$ se, e somente se, $q_*(\pi_1(E, e_0)) \triangleleft \pi_1(X, x_0)$. Nesse caso: $$D \cong \frac{\pi_1(X, x_0)}{q_*(\pi_1(E, e_0))}$$
\end{thm}

\begin{titlemize}{Lista de consequências}
	\item \hyperref[g-recobrimentos-e-epimorfismos-prop]{G-recobrimentos 0-conexos e epimorfismos do grupo fundamental em G};
\end{titlemize}

%---------------------------------------------------------------------------------------------------------------------!Draft!-----------------------------------------------------------------------------------------------------------------
\subsection{G-recobrimento regular}
\label{g-recobrimento-regular-def}
\begin{titlemize}{Lista de dependências}
	\item \hyperref[automorfismo-de-recobrimento-def]{Automorfismo de um recobrimento};
\end{titlemize}
\begin{defi}[G-recobrimento regular]
	Um G-recobrimento regular, ou recobrimento G-regular ou ainda recobrimento G-normal de X é um recobrimento $q:E \longrightarrow X$ tal que $Aut(E, q, X) = G$, e a ação $G \circlearrowright q^{-1}(x)$ é livre e transitiva $\forall x \in X$.
\end{defi}

\begin{titlemize}{Lista de consequências}
	\item \hyperref[g-recobrimentos-e-epimorfismos-prop]{G-recobrimentos 0-conexos e epimorfismos do grupo fundamental em G};
\end{titlemize}

\subsection{G-recobrimentos 0-conexos e epimorfismos do grupo fundamental em G}
\label{g-recobrimentos-e-epimorfismos-prop}
\begin{titlemize}{Lista de dependências}
    \item \hyperref[recobrimento-1-conexo-em-bijecao-com-P(X,x)]{Recobrimento 1-conexo em bijeção com $P(X, x_0)$};\\
	\item \hyperref[g-recobrimento-regular-def]{G-recobrimento regular};\\
    \item \hyperref[acao-de-automorfismos-e-livre-prop]{A ação do grupo de automorfismos é livre sobre as fibras};\\
    \item \hyperref[acao-de-automorfismo-transitiva-prop]{Quando a ação do grupo de automorfismos é transitiva sobre as fibras?};
\end{titlemize}

Antes de enunciar o teorema, lembramos que um modelo conjuntista para o recobrimento universal $\tilde X$ é o conjunto $\{[\gamma] \; | \; \gamma:I \longrightarrow X \; e \; \gamma (0) = x_0\}$, e que $p: \tilde X \longrightarrow X$ é dada por $p([\gamma]) = \gamma(1)$.

\begin{thm}[G-recobrimentos 0-conexos e epimorfismos do grupo fundamental em G]
    Sejam $Epi(\pi_1(X, x_0), G)$ o conjunto dos epimorfismos do grupo fundamental de $X$ sobre um grupo $G$ e seja $Cov_G^0(X, x_0)$ o conjunto dos G-recobrimentos regulares pontuados 0-conexos de X, então existe uma correspondência biunívoca entre esses conjuntos a menos de um único homeomorfismo entre os recobrimentos.
\end{thm}

\begin{dem}
    Primeiramente tome $\phi \in Epi(\pi_1(X, x_0), G)$, defina $E_{\phi} = \frac{(\tilde X, [c_{x_0}])}{Ker(\phi)}$ e $e_0 = \overline{[c_{x_0}]}$. Defina ainda $q:(E_{\phi}, e_0) \longrightarrow (X, x_0)$ por: $$q(\overline{[\gamma]}) = p([\gamma]) = \gamma(1).$$

    Note que $q$ está bem definida, pois dados $\overline{[\gamma]} = \overline{[\gamma']}$, então existe $[\alpha] \in Ker(\phi)$ tal que: $$[\gamma'] = [\alpha] \cdot [\gamma] = [\gamma * \alpha]$$ de modo que: $$q([\gamma']) = \gamma'(1) = (\gamma * \alpha)(1) = \gamma(1) = q([\gamma])$$

    Como $p: (\tilde X, \tilde x_0) \longrightarrow (X, x_0)$ é um recobrimento e $q(\overline{[\gamma]}) = p([\gamma])$,  então $q:(E_{\phi}, e_0) \longrightarrow (X, x_0)$ é recobrimento. Além disso, como $\tilde X$ é $1$-conexo, então dados $[\gamma], [\gamma'] \in \tilde X$, temos que existe $\Gamma: I \longrightarrow \tilde X$ tal que $\Gamma(0) = [\gamma]$ e $\Gamma(1) = [\gamma']$. Defina $\overline{\Gamma}: I \longrightarrow E_{\phi}$ por $\overline{\Gamma}(t) = \overline{\Gamma(t)}$, então $\overline{\Gamma}(0) = \overline{[\gamma]}$ e $\overline{\Gamma}(1) = \overline{[\gamma']}$ e segue que $E_{\phi}$ é $0$-conexo.

    Pelo teorema dos isomorfismos, sabemos que $ker(\phi) \triangleleft \pi_1(X, x_0)$ e $G \cong \frac{\pi_1(X, x_0)}{Ker(\phi)}$. Notando que $Ker(\phi) = q_*(\pi_1(E_{\phi}, e_0))$, segue que $G \cong Aut(E_{\phi}, q, e_0)$ e o recobrimento é G-regular, concluindo a primeira parte da demonstração.

    Em sequência, tome um G-recobrimento $0$-conexo regular pontuado $q:(E, e_0) \longrightarrow (X, x_0)$ e defina $\phi_{(E, e_0)}: \pi_1(X, x_0) \longrightarrow G$ onde $\phi_{(E, e_0)}([\alpha]) = g$ quando $\tilde \alpha_{e_0}(1) = g \cdot e_0$, já vimos que esta função está bem definida e é um epimorfismo em \hyperref[acao-de-automorfismo-transitiva-prop]{"Quando a ação do grupo de automorfismos é transitiva sobre as fibras?"}, concluindo a segunda parte da demonstração.

    Por fim, sejam $\phi_{(E, e_0)} = \phi_{(E', e_0')}$ e $K = Ker(\phi_{(E, e_0)}) = Ker(\phi_{(E', e_0')})$, já sabemos que $(E, e_0) \cong (\frac{\tilde X}{K}, \overline{[c_{x_0}]}) \cong (E', e_0')$, de modo que o seguinte diagrama comuta:

    % https://q.uiver.app/#q=WzAsNCxbMiwwLCIoXFxmcmFje1xcdGlsZGUgWH17S30sIFxcb3ZlcmxpbmV7W2Nfe3hfMH1dfSkiXSxbNCwwLCIoRScsIGVfMCcpIl0sWzAsMCwiKEUsZV8wKSJdLFsyLDMsIlgiXSxbMCwyLCJcXGNvbmciLDJdLFswLDEsIlxcY29uZyJdLFsxLDMsInEnIl0sWzAsMywiXFx0aWxkZSBxIl0sWzIsMywicSIsMl1d
    \[\begin{tikzcd}
    	{(E,e_0)} && {(\frac{\tilde X}{K}, \overline{[c_{x_0}]})} && {(E', e_0')} \\
    	\\
    	\\
    	&& X
    	\arrow["q"', from=1-1, to=4-3]
    	\arrow["\cong"', from=1-3, to=1-1]
    	\arrow["\cong", from=1-3, to=1-5]
    	\arrow["{\tilde q}", from=1-3, to=4-3]
    	\arrow["{q'}", from=1-5, to=4-3]
    \end{tikzcd}\]

    Logo, ambos os homeomorfismos são levantamentos de $\tilde q$, de forma que o contexto de espaços pontuados nos garante suas unicidades. Portanto  , existe um único homeomorfismo entre $(E, e_0)$ e $(E', e_0')$.
\end{dem}

\begin{titlemize}{Lista de consequências}
	\item \hyperref[homomorfismos-e-g-recobrimentos-prop]{G-recobrimentos e homomorfismos do grupo fundamental em G};
\end{titlemize}

