\section{Teorema de Seifert-Van Kampen}
\label{teorema-de-seifert-van-kampen}

\begin{titlemize}{Lista de Dependências}
	\item \hyperref[grupo-fundamental]{Grupo Fundamental};\\
	\item \hyperref[espaco-de-recobrimento]{Espaço de recobrimento};\\
    	\item \hyperref[ações-de-grupos-e-recobrimentos]{Ações de grupos e recobrimentos};\\
    	\item \hyperref[recobrimento-universal]{Recobrimento Universal};
\end{titlemize}

O teorema de Seifert-Van Kampen é um importante resultado da topologia algébrica que nos permite calcular o grupo fundamental de diversos espaços topológicos razoáveis a partir dos grupos fundamentais de certos subespaços.

A menos que explicitamente mencionado, no que segue assumiremos que o espaço base $X$ é razoável, isto é, $X$ é conexo, localmente conexo por caminhos, e localmente semi-simplesmente conexo. Como visto em seções anteriores, $X$ conexo e localmente conexo por caminhos nos garante a exisência e unicidade de levantamentos, $X$ localmente semi-simplesmente conexo nos garante a existência do recobrimento universal, que será denotado por $p:(\tilde X, \tilde x_0) \longrightarrow (X, x_0)$.
\subsection{Automorfismo de um recobrimento}
\label{automorfismo-de-recobrimento-def}
\begin{titlemize}{Lista de dependências}
	\item \hyperref[ações-de-grupo-def]{Ações de grupo};
\end{titlemize}
\begin{defi}[Automorfismo de um recobrimento]
    Um automorfismo de um recobrimento $q:E \longrightarrow X$ é um homeomorfismo $\phi:E \longrightarrow E$ tal que:
    \[\begin{tikzcd}
	E && E \\
	\\
	& X
	\arrow["\phi", from=1-1, to=1-3]
	\arrow["q"', from=1-1, to=3-2]
	\arrow["q", from=1-3, to=3-2]
    \end{tikzcd}\]
\end{defi}

Denotamos o grupo formado por todos os automorfismos de um recobrimento, também chamado de grupo de transformações de Deck, por $Aut(E, q, X)$. Por definição, para todo $x \in X$, a ação $D \circlearrowright q^{-1}(x)$ é dada por $$\phi \cdot e = \phi(e)$$

\begin{titlemize}{Lista de consequências}
	\item \hyperref[acao-de-automorfismos-e-livre-prop]{A ação do grupo de automorfismos é livre};\\
    \item \hyperref[acao-de-automorfismo-transitiva-prop]{Quando a ação do grupo de automorfismos é transitiva sobre as fibras?};\\
    \item \hyperref[g-recobrimento-regular-def]{G-recobrimento regular};
\end{titlemize}
\subsection{A açào do grupo de automorfismos é livre}
\label{acao-de-automorfismos-e-livre-prop}
\begin{titlemize}{Lista de dependências}
	\item \hyperref[levantamento-de-funções-prop]{Levantamento de funções};\\
	\item \hyperref[automorfismo-de-recobrimento-def]{Automorfismo de um recobrimento};
\end{titlemize}
\begin{prop}[Nome da Afirmação]
	Dado um recobrimento $q:E \longrightarrow X$ $0$-conexo, então a ação $Aut(E, q, X) \circlearrowright q^{-1}(x)$ é livre para todo $x \in X$. Ou seja, $\forall x \in X$ $\forall e \in q^{-1}(x)$, $\phi \cdot e = e \Longrightarrow \phi = Id$.\\
\end{prop}

\begin{dem}
    Note que $\phi$ é um levantamento de $q$, logo, como $E$ é conexo e $Id:E \longrightarrow E$ também é um levantamento de q, pela unicidade de levantamentos, segue que dados $x \in X$ e $e \in q^{-1}(x)$, então $$\phi \cdot e = e \Longrightarrow \phi(e) = Id(e) \Longrightarrow \phi = Id$$
    % https://q.uiver.app/#q=WzAsNCxbMiwwLCJFIl0sWzIsMiwiWCJdLFswLDIsIkUiXSxbMSw0LCJcXGJ1bGxldCJdLFsyLDAsIklkIiwyXSxbMiwwLCJcXHBoaSIsMCx7Im9mZnNldCI6LTN9XSxbMCwxLCJwIiwyXSxbMiwxLCJwIiwyXV0=
    \[\begin{tikzcd}
    	&& E \\
    	\\
    	E && X \\
    	\arrow["q"', from=1-3, to=3-3]
    	\arrow["Id"', from=3-1, to=1-3]
    	\arrow["\phi", shift left=3, from=3-1, to=1-3]
    	\arrow["q"', from=3-1, to=3-3]
    \end{tikzcd}\]
\end{dem}

\begin{titlemize}{Lista de consequências}
	\item \hyperref[acao-de-automorfismo-transitiva-prop]{Quando a ação do grupo de auitomorfismos é transitiva sobre as fibras?};
\end{titlemize}
\subsection{Quando a ação do grupo de automorfismos é transitiva sobre as fibras?}
\label{acao-de-automorfismo-transitiva-prop}
\begin{titlemize}{Lista de dependências}
    \item \hyperref[levantamento-de-caminhos-prop]{Levantamento de caminhos};\\
	\item \hyperref[automorfismo-de-recobrimento-def]{Automorfismo de um recobrimento};\\
    \item \hyperref[acao-de-automorfismos-e-livre-prop]{A ação do grupo de automorfismos é livre sobre as fibras};
\end{titlemize}
Seja $q:(E, e_0) \longrightarrow (X, x_0)$ um recobrimento 0-conexo pontuado e $D=Aut(E, q, X)$ o grupo de automorfismos desse recobrimento, queremos achar alguma condição necessária e suficiente para que a ação de $D$ seja transitiva sobre as fibras, ou seja, $\forall x \in X$ a ação $D \circlearrowright q^{-1}(x)$ é tal que $\forall e, e' \in q^{-1}(x)$ $\exists \phi \in D$ tal que $\phi(e) = e'$.

Primeiramente, suponha que a ação $D \circlearrowright q^{-1}(x_0)$ é transitiva. Considere a função $\varphi_{e_0}:\pi_1(X, x_0) \longrightarrow D$, onde $\varphi_{e_0}([\alpha]) = d \Longleftrightarrow d \cdot e_0 = \tilde \alpha_{e_0}(1)$.

\begin{af}
    A função $\varphi_{e_0}$ acima está bem definida e é um homomorfismo sobrejetor.
\end{af}

\begin{dem}
    Tome $[\alpha] \in \pi_1(X, x_0)$, como $\alpha(1) = x_0$, segue que $\tilde\alpha_{e_0}(1) \in q^{-1}(x_0)$, temos ainda que $e_0 \in q^{-1}(x_0)$, logo, como a ação $D \circlearrowright q^{-1}(x_0)$ é transitiva por hipótese, seque que $\exists d \in D$ tal que $d \cdot e_0 = \tilde\alpha_{e_0}(1)$.
    
    Vimos ainda que a ação $D \circlearrowright q^{-1}(x_0)$ é livre, donde segue que $d$ é único. Por fim, se $\alpha \sim \alpha'$ $rel$ $\partial I$, segue que $\tilde\alpha_{e_0}(1) = \tilde\alpha_{e_0}'(1)$, logo $\varphi_{e_0}$ não depende da escolha de representantes e está bem definida.

    Tome $d \in D$, como $E$ é $0$-conexo, $\exists \gamma:I \longrightarrow E$ tal que $\gamma(0) = e_0$ e $\gamma(1) = d \cdot e_0$. Seja $\alpha := q \circ \gamma$, então por unicidade de levantamentos segue que $\tilde\alpha_{e_0} = \gamma$, donde $\varphi_{e_0}([\alpha]) = d$ e $\varphi_{e_0}$ é sobrejetora.

    Por fim, tome $[\alpha], [\beta] \in \pi_1(X, x_0)$, então: $$\varphi_{e_0}([\alpha] \cdot [\beta]) = \varphi([\alpha*\beta])$$ e temos que: $$\widetilde{(\alpha * \beta)}_{e_0}(1) = (\tilde\alpha_{e_0}*\tilde\beta_{\tilde\alpha_{e_0}(1)})(1) = \tilde\beta_{\tilde\alpha_{e_0}(1)}(1).$$

    Por outro lado, sejam $$d'e_0 = \tilde\alpha_{e_0}(1)$$ $$d''e_0 = \tilde\beta_{e_0}(1),$$ então veja que: $$\varphi_{e_0}([\alpha]) \cdot \varphi_{e_0}([\beta]) = d'd''$$ e que $$\tilde\beta_{\tilde\alpha_{e_0}(1)} = \tilde\beta_{d'e_0}$$

    Note que: $$q \circ (d'\tilde\beta_{e_0}) = q \circ d' \circ \tilde\beta_{e_0} = q \circ \tilde\beta_{e_0} = \beta,$$ pois $d'$ é um automorfismo de recobrimento. Além disso: $$d'\tilde\beta_{e_0}(0) = d'e_0$$ de modo que, pela unicidade de levantamentos: $$\tilde\beta_{d'e_0} = d'\tilde\beta_{e_0}$$

    Disso segue que: $$\widetilde{(\alpha * \beta)}_{e_0}(1) = \tilde\beta_{d'e_0}(1) = d'\tilde\beta_{e_0}(1) = d'd''e_0$$ e portanto: $$\varphi([\alpha]\cdot[\beta]) = d'd'' = \varphi_{e_0}([\alpha]) \cdot \varphi_{e_0}([\beta]).$$
\end{dem}

\begin{prop}
    O kernel de $\varphi_{e_0}$ é $q_*(\pi_1(E, e_0))$.
\end{prop}

\begin{dem}
Tome $[\alpha] \in Ker(\varphi_{e_0})$, então:\\
    $$\begin{tabular}{l l}
        $[\alpha] \in Ker(\varphi_{e_0})$ & $\Longleftrightarrow                                                     \tilde\alpha_{e_0}(1) = e_0$ \\
                                        & $\Longleftrightarrow \tilde\alpha_{e_0} \in \Omega(E, e_0)$\\
                                        & $\Longleftrightarrow [\tilde\alpha_{e_0}] \in \pi_1(E, e_0)$\\
                                        & $\Longleftrightarrow [\alpha] \in Im(q_*)$
    \end{tabular}$$\\
    de modo que $Ker(\varphi_{e_0}) = q_*(\pi_1(E, e_0))$.
\end{dem}

Note que isso implica que $q_*(\pi_1(E, e_0)) \triangleleft \pi_1(X, x_0)$ e, ainda: $$D \cong \frac{\pi_1(X, x_0)}{q_*(\pi_1(E, e_0))}$$ pelo teorema de isomorfismos.

Reciprocamente, supondo que $q_*(\pi_1(E, e_0)) \triangleleft \pi_1(X, x_0)$, veremos que $D \cong \frac{\pi_1(X, x_0)}{q_*(\pi_1(E, e_0))}$ e age transitivamente em $q^{-1}(x_0)$.

\begin{af}
    Nas condições acima, se $D$ age transitivamente em $q^{-1}(x_0)$, então age transitivamente em todas as fibras.
\end{af}

\begin{dem}
    Como $X$ é conexo por caminhos, segue que $\pi_1(X, x_0) \cong \pi_1(X, x_1)$ $\forall x_1 \in X$, analogamente $\pi_1(E, e_0) \cong \pi_1(E, e_1)$ $\forall e_1 \in E$. Logo, dados $x_1 \in X$ e $e_1 \in E$ temos $\frac{\pi_1(X, x_0)}{q_*(\pi_1(E, e_0))} \cong D \cong \frac{\pi_1(X, x_1)}{q_*(\pi_1(E, e_1))}$ e, como $\frac{\pi_1(X, x_1)}{q_*(\pi_1(E, e_1))}$ age transitivamente em $q^{-1}(x_1)$, da arbitrariedade de $x_1$ segue que $D$ age transitivamente em todas as fibras.
\end{dem}

\begin{lemma}
    Seja $G$ um grupo e suponha que a ação $G \circlearrowright E$ é propriamente descontínua, então $D \cong G$.
\end{lemma}

\begin{dem}
    Seja $\psi:G \longrightarrow D$ a função dada por $\psi(g) = \psi_g$, onde $\psi_g$ é o homeomorfismo determinado pela ação do elemento g em X, $\psi$ é um homomorfismo por se tratar de uma ação.

    Primeiro note que $\psi$ é injetor, afinal, se $\psi_g = Id$, então $g \cdot e = e$, de modo que $g = 1_G$, pois toda ação propriamente descontínua é livre. Além disso, $\psi$ é sobrejetor, afinal dado $\phi \in D$, sabemos que $$q(e_0) = (q \circ \phi)(e_0)$$ pois $\phi$ é automorfismo de recobrimentos. Portanto, $\exists g \in G$ tal que $\phi(e_0) = ge_0$. Logo, como ambos $\psi_g$ e $\phi$ sào levantamentos de $q$, segue da unicidade de levantamentos que $\psi_g = \phi$.
\end{dem}

\begin{af}
    Vale que $(E, e_0) \cong (\frac{\tilde X}{q_*(\pi_1(E, e_0))}, \overline{[c_{x_0}]})$
\end{af}

\begin{af}
    Seja $G = \frac{\pi_1(X, x_0)}{q_*(\pi_1(E, e_0))}$, então a ação $G \circlearrowright E$ dada por $$\overline{[\alpha]} \cdot \overline{[\gamma]} = \overline{[\alpha * \gamma]}$$ é propriamente descontínua com quociente $X$.
\end{af}

\begin{thm}[Condição necessária e suficiente para a ação do grupo de automorfismos ser transitiva sobre as fibras]
	Se $q:E \longrightarrow X$ é um recobrimento $0$-conexo, então a ação $D \circlearrowright q^{-1}(x)$ é transitiva para todo $x \in X$ se, e somente se, $q_*(\pi_1(E, e_0)) \triangleleft \pi_1(X, x_0)$. Nesse caso: $$D \cong \frac{\pi_1(X, x_0)}{q_*(\pi_1(E, e_0))}$$
\end{thm}

\begin{titlemize}{Lista de consequências}
	\item \hyperref[g-recobrimentos-e-epimorfismos-prop]{G-recobrimentos 0-conexos e epimorfismos do grupo fundamental em G};
\end{titlemize}

%---------------------------------------------------------------------------------------------------------------------!Draft!-----------------------------------------------------------------------------------------------------------------
\subsection{G-recobrimento regular}
\label{g-recobrimento-regular-def}
\begin{titlemize}{Lista de dependências}
	\item \hyperref[automorfismo-de-recobrimento-def]{Automorfismo de um recobrimento};
\end{titlemize}
\begin{defi}[G-recobrimento regular]
	Um G-recobrimento regular, ou recobrimento G-regular ou ainda recobrimento G-normal de X é um recobrimento $q:E \longrightarrow X$ tal que $Aut(E, q, X) = G$, e a ação $G \circlearrowright q^{-1}(x)$ é livre e transitiva $\forall x \in X$.
\end{defi}

\begin{titlemize}{Lista de consequências}
	\item \hyperref[g-recobrimentos-e-epimorfismos-prop]{G-recobrimentos 0-conexos e epimorfismos do grupo fundamental em G};
\end{titlemize}

\subsection{G-recobrimentos 0-conexos e epimorfismos do grupo fundamental em G}
\label{g-recobrimentos-e-epimorfismos-prop}
\begin{titlemize}{Lista de dependências}
    \item \hyperref[recobrimento-1-conexo-em-bijecao-com-P(X,x)]{Recobrimento 1-conexo em bijeção com $P(X, x_0)$};\\
	\item \hyperref[g-recobrimento-regular-def]{G-recobrimento regular};\\
    \item \hyperref[acao-de-automorfismos-e-livre-prop]{A ação do grupo de automorfismos é livre sobre as fibras};\\
    \item \hyperref[acao-de-automorfismo-transitiva-prop]{Quando a ação do grupo de automorfismos é transitiva sobre as fibras?};
\end{titlemize}

Antes de enunciar o teorema, lembramos que um modelo conjuntista para o recobrimento universal $\tilde X$ é o conjunto $\{[\gamma] \; | \; \gamma:I \longrightarrow X \; e \; \gamma (0) = x_0\}$, e que $p: \tilde X \longrightarrow X$ é dada por $p([\gamma]) = \gamma(1)$.

\begin{thm}[G-recobrimentos 0-conexos e epimorfismos do grupo fundamental em G]
    Sejam $Epi(\pi_1(X, x_0), G)$ o conjunto dos epimorfismos do grupo fundamental de $X$ sobre um grupo $G$ e seja $Cov_G^0(X, x_0)$ o conjunto dos G-recobrimentos regulares pontuados 0-conexos de X, então existe uma correspondência biunívoca entre esses conjuntos a menos de um único homeomorfismo entre os recobrimentos.
\end{thm}

\begin{dem}
    Primeiramente tome $\phi \in Epi(\pi_1(X, x_0), G)$, defina $E_{\phi} = \frac{(\tilde X, [c_{x_0}])}{Ker(\phi)}$ e $e_0 = \overline{[c_{x_0}]}$. Defina ainda $q:(E_{\phi}, e_0) \longrightarrow (X, x_0)$ por: $$q(\overline{[\gamma]}) = p([\gamma]) = \gamma(1).$$

    Note que $q$ está bem definida, pois dados $\overline{[\gamma]} = \overline{[\gamma']}$, então existe $[\alpha] \in Ker(\phi)$ tal que: $$[\gamma'] = [\alpha] \cdot [\gamma] = [\gamma * \alpha]$$ de modo que: $$q([\gamma']) = \gamma'(1) = (\gamma * \alpha)(1) = \gamma(1) = q([\gamma])$$

    Como $p: (\tilde X, \tilde x_0) \longrightarrow (X, x_0)$ é um recobrimento e $q(\overline{[\gamma]}) = p([\gamma])$,  então $q:(E_{\phi}, e_0) \longrightarrow (X, x_0)$ é recobrimento. Além disso, como $\tilde X$ é $1$-conexo, então dados $[\gamma], [\gamma'] \in \tilde X$, temos que existe $\Gamma: I \longrightarrow \tilde X$ tal que $\Gamma(0) = [\gamma]$ e $\Gamma(1) = [\gamma']$. Defina $\overline{\Gamma}: I \longrightarrow E_{\phi}$ por $\overline{\Gamma}(t) = \overline{\Gamma(t)}$, então $\overline{\Gamma}(0) = \overline{[\gamma]}$ e $\overline{\Gamma}(1) = \overline{[\gamma']}$ e segue que $E_{\phi}$ é $0$-conexo.

    Pelo teorema dos isomorfismos, sabemos que $ker(\phi) \triangleleft \pi_1(X, x_0)$ e $G \cong \frac{\pi_1(X, x_0)}{Ker(\phi)}$. Notando que $Ker(\phi) = q_*(\pi_1(E_{\phi}, e_0))$, segue que $G \cong Aut(E_{\phi}, q, e_0)$ e o recobrimento é G-regular, concluindo a primeira parte da demonstração.

    Em sequência, tome um G-recobrimento $0$-conexo regular pontuado $q:(E, e_0) \longrightarrow (X, x_0)$ e defina $\phi_{(E, e_0)}: \pi_1(X, x_0) \longrightarrow G$ onde $\phi_{(E, e_0)}([\alpha]) = g$ quando $\tilde \alpha_{e_0}(1) = g \cdot e_0$, já vimos que esta função está bem definida e é um epimorfismo em \hyperref[acao-de-automorfismo-transitiva-prop]{"Quando a ação do grupo de automorfismos é transitiva sobre as fibras?"}, concluindo a segunda parte da demonstração.

    Por fim, sejam $\phi_{(E, e_0)} = \phi_{(E', e_0')}$ e $K = Ker(\phi_{(E, e_0)}) = Ker(\phi_{(E', e_0')})$, já sabemos que $(E, e_0) \cong (\frac{\tilde X}{K}, \overline{[c_{x_0}]}) \cong (E', e_0')$, de modo que o seguinte diagrama comuta:

    % https://q.uiver.app/#q=WzAsNCxbMiwwLCIoXFxmcmFje1xcdGlsZGUgWH17S30sIFxcb3ZlcmxpbmV7W2Nfe3hfMH1dfSkiXSxbNCwwLCIoRScsIGVfMCcpIl0sWzAsMCwiKEUsZV8wKSJdLFsyLDMsIlgiXSxbMCwyLCJcXGNvbmciLDJdLFswLDEsIlxcY29uZyJdLFsxLDMsInEnIl0sWzAsMywiXFx0aWxkZSBxIl0sWzIsMywicSIsMl1d
    \[\begin{tikzcd}
    	{(E,e_0)} && {(\frac{\tilde X}{K}, \overline{[c_{x_0}]})} && {(E', e_0')} \\
    	\\
    	\\
    	&& X
    	\arrow["q"', from=1-1, to=4-3]
    	\arrow["\cong"', from=1-3, to=1-1]
    	\arrow["\cong", from=1-3, to=1-5]
    	\arrow["{\tilde q}", from=1-3, to=4-3]
    	\arrow["{q'}", from=1-5, to=4-3]
    \end{tikzcd}\]

    Logo, ambos os homeomorfismos são levantamentos de $\tilde q$, de forma que o contexto de espaços pontuados nos garante suas unicidades. Portanto  , existe um único homeomorfismo entre $(E, e_0)$ e $(E', e_0')$.
\end{dem}

\begin{titlemize}{Lista de consequências}
	\item \hyperref[homomorfismos-e-g-recobrimentos-prop]{G-recobrimentos e homomorfismos do grupo fundamental em G};
\end{titlemize}

\subsection{G-recobrimentos e homomorfismos do grupo fundamental em G}
\label{homomorfismos-e-g-recobrimentos-prop}
\begin{titlemize}{Lista de dependências}
    \item \hyperref[acao-de-automorfismos-e-livre-prop]{A ação do grupo de automorfismos é livre sobre as fibras};\\
    \item \hyperref[acao-de-automorfismo-transitiva-prop]{Quando a ação do grupo de automorfismos é transitiva sobre as fibras?};\\
    \item \hyperref[g-recobrimento-regular-def]{G-recobrimento regular};\\
	\item \hyperref[g-recobrimentos-e-epimorfismos-prop]{G-recobrimentos 0-conexos e epimorfismos do grupo fundamental em G};
\end{titlemize}

Em \hyperref[g-recobrimentos-e-epimorfismos-prop]{"G-recobrimentos 0-conexos e epimorfismos do grupo fundamental em G"} vimos que para todo $G$-recobrimento regular $0$-conexo $q:(E, e_0) \longrightarrow (X, x_0)$ existe um epimorfismo do grupo fundamental $\pi_1(X, x_0)$ em $G$. Ainda, se $K$ é o kernel desse epimorfismo, então $G \cong \frac{\pi_1(X, x_0)}{K}$ e $E$ é o quociente de $\tilde X$ pela ação de $K$, com $e_0$ sendo a imagem de $\tilde x_0$. Queremos estender essa correspondência para $G$-recobrimentos não necessariamente $0$-conexos, nesse caso obteremos homomorfismos de $\pi_1(X, x_0)$ em $G$ não necessariamente sobrejetores.

Suponha que $\phi: \pi_1(X, x_0) \longrightarrow G$ é um homomorfismo para algum grupo $G$. De a $G$ a topologia discreta, então o produto cartesiano $\tilde X \times G$ é um produto de cópias de $\tilde X$, uma para cade elemento de $G$. O grupo $\pi_1(X, x_0)$ age em $\tilde X \times G$ segundo a regra: $$[\alpha] \cdot (\tilde x, g) = ([\alpha] \cdot \tilde x, g \cdot \phi([\alpha]^{-1})),$$ onde $[\alpha] \in \pi_1(X, x_0)$, $\tilde x \in \tilde X$ e $g \in G$. Aqui, $[\alpha] \cdot \tilde x$ representa a ação de $\pi_1(X, x_0)$ sobre $\tilde X$ e $g \cdot \phi([\alpha]^{-1})$ é o produto em $G$.

Defina $E_{\phi}$ como o quociente de $\tilde X \times G$ por essa ação: $$E_{\phi} = \frac{\tilde X \times G}{\pi_1(X, x_0)}$$ e seja $e_0$ e imagem de $(\tilde x_0, e)$. Denote por $\langle \tilde x, g \rangle$ a imagem em $E_{\phi}$ de $(\tilde x, g)$. Note que, dada ação de $\pi_1(X, x_0)$ em $\tilde X \times G$, para $\tilde x \in \tilde X$, $g \in G$ e $[\alpha] \in \pi_1(X, x_0)$, temos: $$\langle [\alpha] \cdot (\tilde x, g) \rangle = \langle \tilde x, g \cdot \phi([\alpha]) \rangle.$$ Defina $q_{\phi}: E_{\phi} \longrightarrow X$ por $q_{\phi}(\langle \tilde x, g \rangle) = p(\tilde x)$. Note que por $p: \tilde X \longrightarrow X$ ser um recobrimento, essa definição nos garante que $q_{\phi}:E_{\phi} \longrightarrow X$ também é um recobrimento.

O grupo $G$ age em $E_{\phi}$ pela fórmula $h \cdot \langle \tilde x, g \rangle = \langle \tilde x, h \cdot g \rangle$, para $g, h \in G$ e $\tilde x \in \tilde X$. Note que usamos o lado direito de $G$ para a ação de $\pi_1(X, x_0)$, de modo que o lado esquerdo de $G$ ficou livre para a ação de $G$. Veremos a seguir que essa é uma ação propriamente descontínua de modo que, pelos resultados apresentados em \hyperref[acao-de-automorfismos-e-livre-prop]{"A ação do grupo de automorfismos é livre sobre as fibras"} e \hyperref[acao-de-automorfismo-transitiva-prop]{"Quando a ação do grupo de automorfismos é transitiva sobre as fibras?"}, o recobrimento se trata de um G-recobrimento.

Seja $N$ qualquer subconjunto de $X$ sobre o qual o recobrimento universal é trivial, então existe um isomorfismo de $p^{-1}(N)$ com o recobrimento produto $N \times \pi_1(X, x_0)$, onde $\pi_1(X, x_0)$ age a esquerda no segundo fator. Isso nos dá homeomorfismos: $$q_{\phi}^{-1}(N) \cong (N \times \pi_1(X, x_0)) \times \frac{G}{\pi_1(X, x_0)} \cong N \times G,$$ onde o último homeomorfismo é dado por $\langle (n, [\alpha]), g \rangle \mapsto (n, g \cdot \phi([\alpha]))$, com o mapa inverso sendo $(n, g) \mapsto \langle (n, e), g \rangle$. Esse homeomorfismos são compatíveis com as projeções para $N$, donde segue que, sobre $N$, a ação de $G$ é propriamente descontínua. Como $X$ é coberto por conjuntos abertos como $N$, o mesmo é verdade para o mapa $q_{\phi}:E_{\phi} \longrightarrow X$.

Por outro lado, suponha que $q: (E, e_0) \longrightarrow (X, x_0)$ é um $G$-recobrimento regular pontuado, construiremos um homomorfismo $\phi:\pi_1(X, x_0) \longrightarrow G$. Para cada $[\alpha] \in \pi_1(X, x_0)$, o elemento $\phi([\alpha]) \in G$ será determinado por: $$\phi([\alpha]) = g \Longleftrightarrow g \cdot e_0 = \tilde \alpha_{e_0}(1).$$ Em \hyperref[acao-de-automorfismo-transitiva-prop]{"Quando a ação do grupo de automorfismos é transitiva sobre as fibras?"}, sob a hipótese adicional de $E$ ser $0$-conexo, vimos que esse mapa está bem definido e é um epimorfismo. Notando que a hipótese adicional foi utilizada apenas para mostrar que $\phi$ é sobrejetor, segue que $\phi$ está bem definido e é um homomorfismo.

\begin{thm}[Correspondência entre G-recobrimentos e homomorfismos do grupo fundamental em G]
	As construções acima determinam uma correspondência biunívoca entre o conjunto dos homomorfismos de $\pi_1(X, x_0)$ em $G$ e o conjunto de $G$-recobrimentos regulares pontuados, a menos de isomorfismo.
\end{thm}

\begin{dem}
    Dado um G-recobrimento regular $q:(E, e_0) \longrightarrow (X, x_0)$ e o homomorfismo $\phi: \pi_1(X, x_0) \longrightarrow G$ construído a partir dele, precisamos checar que este recobrimento é isomorfo ao recobrimento $q_{\phi}:E_{\phi} \longrightarrow X$ construído a partir de $\phi$. Para mapear $E_{\phi}$ em $E$, precisamos mapear $\tilde X \times G$ em $E$ e mostrar que as órbitas de $\pi_1(X, x_0)$ tem as mesmas imagens. Para tanto, identificamos o recobrimento universal $\tilde X$ com o espaço de classes de homotopia de caminhos em $X$ começando em $x_0$, $\tilde X \cong \{[\gamma] \; | \; \gamma:I \longrightarrow X, \; \gamma(0) = x_0\}$. Defina o aplicação contínua $\tilde X \times G \longrightarrow E$ por: $$([\gamma], g) \longmapsto g \cdot \tilde \gamma_{e_0}(1) = \tilde \gamma_{g \cdot e_0}(1).$$

    Precisamos checar que que o ponto equivalente $(([\alpha]\cdot[\gamma]),(g\cdot\phi([\alpha])^{-1}))$ é levado à mesma imagem. Note que:\\
        $$\begin{tabular}{l l}
            $(g\cdot\phi([\alpha])^{-1}) \cdot \widetilde{(\gamma * \alpha)}_{e_0}(1)$ & $= (g\cdot\phi([\alpha])^{-1}) \cdot \tilde\gamma_{\tilde\alpha_{e_0}(1)}(1)$\\
                 & $= \tilde\gamma_{(g\cdot\phi([\alpha])^{-1}) \cdot \tilde\alpha_{e_0}(1)}(1)$\\
                 & $= \tilde\gamma_{g \cdot \alpha^{-1}_{\alpha_{e_0}}(1)}(1)$\\
                 & $= \tilde\gamma_{g\cdot(\alpha*\alpha^{-1})_{e_0}(1)}(1)$\\
                 & $= \tilde\gamma_{g \cdot e_0}(1).$
        \end{tabular}$$\\

    Como a aplicação leva pontos equivalentes às mesmas imagens, ela da origem a um mapa do quociente $E_{\phi}$ a $E$, que é um mapa entre espaços de recobrimento de $X$. É fácil checar que se trata de um mapa entre $G$-recobrimentos regulares, donde segue que se trata de um isomorfismo.

    Por outro lado, partindo de um homomorfismo $\phi$ e do $G$-recobrimento regular $q_{\phi}: E_{\phi} \longrightarrow X$ construído a partir dele, obtemos um novo homomorfismo $\overline{\phi}$, precisamos verificar que $\phi = \overline{\phi}$. Para $[\alpha] \in \pi_1(X, x_0)$:\\
        $$\begin{tabular}{l l}
            $\overline{\phi}([\alpha]) \cdot \langle \tilde x_0, e \rangle$ 
                            & $= \tilde\alpha_{\langle \tilde x_0, e \rangle}(1)$\\
                            & $= \langle \tilde\alpha_{\tilde x_0}(1), e \rangle$\\
                            & $= \langle [\alpha] \cdot \tilde x_0, e \rangle$\\
                            & $= \langle \tilde x_0, e \cdot \phi([\alpha]) \rangle$\\
                            & $= \phi([\alpha]) \cdot \langle \tilde x_0, e \rangle$
        \end{tabular}$$\\
    donde temos que $\phi([\alpha]) = \overline{\phi}([\alpha])$.
\end{dem}

\begin{titlemize}{Lista de consequências}
	\item \hyperref[seifert-van-kampen-prop]{Teorema de Seifert-Van Kampen};
\end{titlemize}
\subsection{Colagem de recobrimentos}
\label{colagem-de-recobrimentos-prop}
\begin{titlemize}{Lista de dependências}
	\item \hyperref[espaco-de-recobrimento]{Espaço de recobrimento};
\end{titlemize}
Suponha que $X$ é uma união de dois conjuntos abertos $U$ e $V$. Um recobrimento de $X$ se restringe a $U$ e $V$ gerando dois recobrimentos isomorfos em $U \cap V$. Queremos realizar o processo inverso.
\begin{thm}[Colagem de recobrimentos] 
	Sejam $q_1: E_1 \longrightarrow U$ e $q_2: E_2 \longrightarrow V$ recobrimentos e seja $\varphi: q_1^{-1}(U \cap V) \longrightarrow q_2^{-1}(U \cap V)$ um isomorfismo de recobrimentos sobre $U \cap V$. Então é possível "colar" os recobrimentos de forma a obter um recobrimento $q:E \longrightarrow X$ junto de dois isomorfismos de recobrimentos $\varphi_1:E_1 \longrightarrow q^{-1}(U)$  e $\varphi_2:E_2 \longrightarrow q^{-1}(V)$ tais que, em $U \cap V$, $\varphi = \varphi_2^{-1} \circ \varphi_1$.
\end{thm}

\begin{dem}
    Podemos construir o conjunto $E$ como o espaço quociente da união disjunta $E_1 \sqcup E_2$ pela relação de equivalência que identifica um ponto $e_1 \in q_1^{-1}(U \cap V)$ com o ponto $\varphi(e_1) \in q_2^{-1}(U \cap v)$. Como $\varphi$ é compatível com mapas para $X$, obtemos um mapa $q:E \longrightarrow X$.

    Como o mapa de $E_1$ para $E$ é um homeomorfismo sobre sua imagem $q^{-1}(U)$, que é aberta em $E$, segue que a restrição de $q$ à imagem inversa de $U$ é isomorfa à $E_1 \longrightarrow U$, similarmente, a restrição sobre a imagem inversa de $V$ é isomorfa à $Y_2 \longrightarrow V$. Disso, segue que $q:E \longrightarrow X$ é um recobrimento.
\end{dem}

\begin{corol}
    Se $q_1:E_1 \longrightarrow U$ e $q_2:E_2 \longrightarrow V$ forem $G$-recobrimentos regulares, então $q:E \longrightarrow X$ tem uma única estrutura de $G$-recobrimento regular tal que os mapas de $E_1$ e $E_2$ comutam com a ação de $G$.
\end{corol}

\begin{titlemize}{Lista de consequências}
	\item \hyperref[seifert-van-kampen-prop]{Teorema de Seifert-Van Kampen};
\end{titlemize}

\subsection{Teorema de Seifert-Van Kampen}
\label{seifert-van-kampen-prop}
\begin{titlemize}{Lista de dependências}
	\item \hyperref[homomorfismos-e-g-recobrimentos-prop]{G-recobrimentos e homomorfismos do grupo fundamental em G};\\
	\item \hyperref[colagem-de-recobrimentos-prop]{Colagem de recobrimentos};
\end{titlemize}

O teorema de Seifert-Van Kampen descreve o grupo fundamental da união de dois espaços em termos do grupo fundamental de cada um desses espaços e de sua intersecção. Seja $X$ um espaço que é a união de dois subespaços abertos $U$ e $V$. Seja $x_0$ um ponto da intersecção $U \cap V$. Temos o seguinte diagrama comutativo de homomorfismos dos grupos fundamentais:

% https://q.uiver.app/#q=WzAsNCxbMCwxLCJcXHBpXzEoVSBcXGNhcCBWLCB4XzApIl0sWzEsMCwiXFxwaV8xKFUsIHhfMCkiXSxbMiwxLCJcXHBpXzEoWCwgeF8wKSJdLFsxLDIsIlxccGlfMShWLCB4XzApIl0sWzMsMiwial97ViwgKn0iXSxbMSwyLCJqX3tVLCAqfSIsMl0sWzAsMSwiaV97VSwgKn0iLDJdLFswLDMsImlfe1YsICp9Il1d
\[\begin{tikzcd}
	& {\pi_1(U, x_0)} \\
	{\pi_1(U \cap V, x_0)} && {\pi_1(X, x_0)} \\
	& {\pi_1(V, x_0)}
	\arrow["{j_{U, *}}"', from=1-2, to=2-3]
	\arrow["{i_{U, *}}"', from=2-1, to=1-2]
	\arrow["{i_{V, *}}", from=2-1, to=3-2]
	\arrow["{j_{V, *}}", from=3-2, to=2-3]
\end{tikzcd}\]
onde os mapas são induzidos pelas inclusões de subespaço.

Iremos determinar o grupo $\pi_1(X, x_0)$ a partir dos outros grupos e dos mapas entre eles a partir de uma propriedade universal. Note que qualquer homomorfismo $h$ de $\pi_1(X, x_0)$ para $G$ determina um par de homomorfismos $h_U = h \circ j_{U, *}$ de $\pi_1(U, x_0)$ para G e $h_V = h \circ j_{V, *}$ de $\pi_1(V, x_0)$ para $G$. Os dois homomorfismos $h_U \circ i_{U, *}$ e $h_V \circ i_{V, *}$ de $\pi_1(U \cap V, x_0)$ para $G$ são o mesmo. O teorema de Seifert-Van Kampen diz que $\pi_1(X, x_0)$ é o grupo "universal" com essa propriedade.

\begin{thm}[Seifert-Van Kampen] 
	Para quaisquer homomorfismos $h_U:\pi_1(U, x_0) \longrightarrow G$ e $h_V:\pi_1(V, x_0) \longrightarrow G$ tais que $h_U \circ i_{U, *} = h_V \circ i_{V, *}$, existe um único homomorfismo: $$h:\pi_1(X, x_0) \longrightarrow G$$ tal que $h \circ j_{U, *} = h_U$ e $h \circ j_{V, *} = h_V$.
\end{thm}

% https://q.uiver.app/#q=WzAsNSxbMCwxLCJOIl0sWzEsMCwiXFx7ZVxcfSJdLFsxLDIsIlxce2VcXH0iXSxbMiwxLCJHIl0sWzQsMSwiSCJdLFsxLDMsImpfMSIsMl0sWzIsMywial8yIl0sWzAsMSwiaV8xIiwyXSxbMCwyLCJpXzIiXSxbMyw0LCJoIiwwLHsic3R5bGUiOnsiYm9keSI6eyJuYW1lIjoiZGFzaGVkIn19fV0sWzEsNCwiaF8xIl0sWzIsNCwiaF8yIl1d
\[\begin{tikzcd}
	& {\{e\}} \\
	N && G && H \\
	& {\{e\}}
	\arrow["{j_1}"', from=1-2, to=2-3]
	\arrow["{h_1}", from=1-2, to=2-5]
	\arrow["{i_1}"', from=2-1, to=1-2]
	\arrow["{i_2}", from=2-1, to=3-2]
	\arrow["h", dashed, from=2-3, to=2-5]
	\arrow["{j_2}", from=3-2, to=2-3]
	\arrow["{h_2}", from=3-2, to=2-5]
\end{tikzcd}\]

\begin{dem}
    Considere um grupo $G$ e um par de homomorfismos $h_U$ e $h_V$ como nas hipóteses do teorema. Pelo resultado em \hyperref[homomorfismos-e-g-recobrimentos-prop]{"G-recobrimentos e homomorfismos do grupo fundamental em G"}, podemos associar aos homomorfismos $h_U$ e $h_V$ $G$-recobrimentos regulares $q_U:(E_U,e_{U, 0}) \longrightarrow (U, x_0)$ e $q_V:(E_V,e_{V, 0}) \longrightarrow (V, x_0)$. Para $k = U, V$, seja $Y_k = q_k^{-1}(U \cap V)$, de forma que $(Y_k, y_{k, 0}) \longrightarrow (U \cap V, x_0)$ é um $G$-recobrimento regular representando o homomorfismo: $$h_k \circ i_{k, *}:  \pi_1(U \cap V, x_0) \longrightarrow \pi_1(k, x_0) \longrightarrow G$$

    Como os homomorfismos $h_U \circ i_{U, *}$ e $h_V \circ i_{V, *}$ são iguais, temos que $(Y_U, y_{U, 0}) \longrightarrow (U \cap V, x_0)$ e $(Y_V, y_{V, 0}) \longrightarrow (U \cap V, x_0)$ são $G$-recobrimentos regulares isomorfos, logo existe um único homeomorfismo $\varphi: (Y_U, y_{U, 0}) \longrightarrow (Y_V, y_{V, 0})$. Pelo visto em \hyperref[colagem-de-recobrimentos-prop]{"Colagem de recobrimentos"}, podemos colar $q_U$ e $q_V$ de forma a obter um G-recobrimento regular $q:(E, e_0) \longrightarrow (X, x_0)$. Usando novamente o resultado de \hyperref[homomorfismos-e-g-recobrimentos-prop]{"G-recobrimentos e homomorfismos do grupo fundamental em G"} vemos que esse recobrimento representa o homomorfismo $h$ desejado. A unicidade de $h$ decorre da unicidade de todos os passos realizados.
\end{dem}

\begin{titlemize}{Lista de consequências}
	\item \hyperref[n-esfera-1-conexa-ex]{A n-esfera é 1-conexa};
\end{titlemize}

\subsection{A n-esfera é 1-conexa para $n \geq 2$}
\label{n-esfera-1-conexa-ex}
\begin{titlemize}{Lista de dependências}
	\item \hyperref[seifert-van-kampen-prop]{Teorena de Seifert-Van Kampen};
\end{titlemize}

Suponha que temos o seguinte diagrama:
% https://q.uiver.app/#q=WzAsNSxbMCwxLCJOIl0sWzEsMCwiXFx7ZVxcfSJdLFsxLDIsIlxce2VcXH0iXSxbMiwxLCJHIl0sWzQsMSwiSCJdLFsxLDMsImpfMSIsMl0sWzIsMywial8yIl0sWzAsMSwiaV8xIiwyXSxbMCwyLCJpXzIiXSxbMyw0LCJoIiwwLHsic3R5bGUiOnsiYm9keSI6eyJuYW1lIjoiZGFzaGVkIn19fV0sWzEsNCwiaF8xIl0sWzIsNCwiaF8yIl1d
\[\begin{tikzcd}
	& {\{e\}} \\
	N && G && H \\
	& {\{e\}}
	\arrow["{j_1}"', from=1-2, to=2-3]
	\arrow["{h_1}", from=1-2, to=2-5]
	\arrow["{i_1}"', from=2-1, to=1-2]
	\arrow["{i_2}", from=2-1, to=3-2]
	\arrow["h", dashed, from=2-3, to=2-5]
	\arrow["{j_2}", from=3-2, to=2-3]
	\arrow["{h_2}"', from=3-2, to=2-5]
\end{tikzcd}\]
onde $(G, j_1, j_2)$ satisfazem a propriedade universal do teorema de Seifert-Van Kampen. Note que para qualquer grupo $G$, os homomorfismos $j_1$ e $j_2$ estão fixos e são dados por $j_1(e) = e_G = j_2(e)$. Note ainda que $$(h \circ j_1)(e) = h(j_1(e)) = h(e_G) = e_H = h_1(e)$$ e da mesma forma $$(h \circ j_2)(e) = h_2(e)$$ para qualquer homomorfismo $h:G\longrightarrow H$, logo, para termos a unicidade de $h$ precisamos que $G$ seja um grupo que tenha apenas um homomorfismo para qualquer outro grupo $H$. Portanto, segue que $G = \{e\}$.

\begin{ex}[O grupo fundamental de $S^n$ para $N \geq 2$]
	Sejam: $$S^n = \{(x_1, \cdots, x_n, t) \in \mathbb{R}^{n+1} \; | \; x_1^2+\cdots+x_n^2+t^2 = 1\}$$ $$U = \{(x_1, \cdots, x_n, t) \in S^n \; | \; t > -1/2\}$$ $$V = \{(x_1, \cdots, x_n, t) \in S^n \; | \; t < 1/2\}$$ $$U \cap V = \{(x_1, \cdots, x_n, t) \in S^n \; | \; -1/2 < t < 1/2\}$$

    Então veja que $U \cong \mathring D^n \cong V$, de modo que $\pi_1(U, x_0) = \pi_1(V, x_0) = \{e\}$ e $U \cap V \cong (-\frac{1}{2}, \frac{1}{2}) \times S^{n-1}$, que é $0$-conexo. Logo, pelo discutido acima, segue que $\pi_1(S^n, x_0) = \{e\}$.
\end{ex}

Disso, vemos que a n-esfera, para $n \geq 2$, é $1$-conexa.
