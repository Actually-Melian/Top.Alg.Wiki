%---------------------------------------------------------------------------------------------------------------------!Draft!-----------------------------------------------------------------------------------------------------------------
\subsection{Ações de grupos e recobrimentos}
\label{ações-de-grupos-e-recobrimentos-ex}
\begin{titlemize}{Lista de dependências}
	\item \hyperref[ações-de-gr-propriamente-descontínuas-def]{Ações de grupos propriamente descontínuas};\\ %'dependencia1' é o label onde o conceito Dependência 1 aparece (--à arrumar um padrão para referencias e labels--) 
	\item \hyperref[ações-de-grupos-def]{Ações de grupos};\\
    \item \hyperref[ações-de-grupos-e-gr-fundamental-prop]{Encontrando o grupo fundamental através de uma ação propriamente descontínua}
% quantas dependências forem necessárias.
\end{titlemize}

\begin{ex}[Ações de grupos propriamente descontínuas]
	\begin{itemize}
        \item[1.] $\mathbb{Z}\circlearrowright\mathbb{R}$\\
            $n\cdot x = n+x \text{\ \ \  é uma ação propriamente descontínua. Além disso,} \\\\
            \mkern-18mu \quad{^{\textstyle \mathbb{R}}\big/_{\textstyle \mathbb{Z}}} \cong S^1$ onde,\\\\
            $[x]\longmapsto e^{2\pi ix}\\\\
            \Rightarrow \Pi_1(S^1 , 1) = \mathbb{Z}$\\
            
	    \item[2.] $\mathbb{Z}^n\circlearrowright\mathbb{R}^n$\\
            $(k_1,\dots, k_n)\cdot(x_1,\dots,x_n) = (k_1+n_1,\dots,k_n+x_n)$ é uma ação propriamente descontínua. Além disso,\\\\
            $\mkern-18mu \quad{^{\textstyle \mathbb{R}^n}\big/_{\textstyle \mathbb{Z}^n}} \cong T^n = S^1\times\dots\times S^1$ onde,\\\\
            $([x_1],\dots,[x_n])\longmapsto (e^{2\pi ix_1},\dots,e^{2\pi ix_n})$\\\\
            $\Rightarrow \Pi_1(T^n , p) = \mathbb{Z}^n$\\

        \item[3.] $\mathbb{Z}_2\circlearrowright S^n$, $n\geq2$\\
            $x\longmapsto x\\
             x\longmapsto x$ \ \ \ \ ação propriamente descontínua. Além disso,\\\\
            $\mkern-18mu \quad{^{\textstyle S^n}\big/_{\textstyle \mathbb{Z}_2}} \cong \mathbb{RP}^n$ onde,\\\\
            $\Rightarrow \Pi_1(\mathbb{RP}^n , p) = \mathbb{Z}_2$\\

        \item[4.] Espaço lenticulares\\\\
        Sejam:\\
        $\mathbb{Z}_n = \{\omega \in \mathbb{C}\mid \omega^n =1\}$\\
        $S^3 =\{(z_1,z_2)\in \mathbb{C}\oplus\mathbb{C}\mid |z_1|^2 + |z_2|^2 = 1\}$\\\\
        Então $\omega\cdot(z_1,z_2) = (\omega z_1,\omega z_2)$ \ \ é uma ação propriamente descontínua. Além disso,\\\\
        $\mkern-18mu \quad{^{\textstyle \mathbb{R}^n}\big/_{\textstyle \mathbb{Z}^n}} \cong L_n$ onde, $L_n$ é uma variedade suave\\\\
        
        $\Rightarrow \Pi_1(L_n , p) = \mathbb{Z}_n$\\     
        
    \end{itemize} 
\end{ex}
 
%\begin{titlemize}{Lista de consequências}
	%\item \hyperref[]{};\\ %'consequencia1' é o label onde o conceito Consequência 1 aparece
	%\item \hyperref[]{}
%\end{titlemize}

