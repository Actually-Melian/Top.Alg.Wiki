%---------------------------------------------------------------------------------------------------------------------!Draft!-----------------------------------------------------------------------------------------------------------------
\subsection{Propriedade 1-conexo} %afirmação aqui significa teorema/proposição/colorário/lema
\label{1-conexo-prop}
\begin{titlemize}{Lista de dependências}
	\item \hyperref[espaço-1-conexo-def]{Espaço 1-conexo};\\ %'dependencia1' é o label onde o conceito Dependência 1 aparece (--à arrumar um padrão para referencias e labels--) 
	\item \hyperref[homotopia-relativa-def]{Homotopia relativa};\\
        \item \hyperref[Produto-concatenacao-def]{Concatenação de caminhos};\\
        \item  \hyperref[produto-bem-definido-prop]{Produto do grupo fundamental}
% quantas dependências forem necessárias.
\end{titlemize}


%Comentário sobre os objetos envolvidos na afirmação.


\begin{prop}% ou af(afirmação)/prop(proposição)/corol(corolário)/lemma(lema)/outros ambientes devem ser definidos no preambulo de Alg.Top-Wiki.tex 
	Seja $E$ um espaço 1-conexo e sejam $\alpha$, $\beta: I\rightarrow E$ curvas com $\alpha(0)=\beta(0)$. Então $\alpha \sim \beta (\text{rel }\partial I)$ se e somente se $\alpha(1)=\beta(1)$\\
\end{prop}
\begin{dem}
    Se $\alpha \sim \beta ~(\text{rel }\partial I)$, então $\alpha(1)=\beta(1)$ pela definição de homotopia relativa, que mantém fixos os pontos de $\partial I$, em particular o $1$.
    Pelo outro lado, se temos $\alpha(1)=\beta(1)$, então a curva fechada $\overline{\alpha}* \beta$ com ponto base $\beta(0)$ é homotópica à curva constante em $\beta(1)$ por conta de $E$ ser 1-conexo. Porém, $\overline{\alpha}*\alpha$ também é homotópico à curva constante em $\alpha(1)=\beta(1)$. Isto é, $\overline{\alpha}* \beta$ é homotópico a $\overline{\alpha}*\alpha$ relativo a $x_1=\alpha(1)=\beta(1)$. Assim, utilizando propriedades de concatenação de caminhos presentes em \ref{produto-bem-definido-prop} temos
   $$[\beta]=[c_{x_1}*\beta]=[(\alpha * \overline{\alpha}) *\beta]=[\alpha*(\overline{\alpha}*\beta)]=[\alpha * c_{x_1}]=[\alpha]$$
e portanto, $\alpha$ e $\beta$ estão na mesma classe de homotopia preservando pontos inicial e final, isto é, $\alpha \sim \beta (\text{rel }\partial I)$
\end{dem}

%Comentários sobre a afirmação.

%\begin{titlemize}{Lista de consequências}
%	\item \hyperref[consequencia1]{Consequência 1};\\ %'consequencia1' é o label onde o conceito Consequência 1 aparece
%	\item \hyperref[]{}
%\end{titlemize}

%[Bianca]: Um arquivo tex pode ter mais de uma afirmação (ou definição, ou exemplo), mas nesse caso cada afirmação deve ter seu próprio label. Dar preferência para agrupar afirmações que dependam entre sí de maneira próxima (um teorema e seu corolário, por exemplo)
