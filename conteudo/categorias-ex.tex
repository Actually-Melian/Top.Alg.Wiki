%---------------------------------------------------------------------------------------------------------------------!Draft!-----------------------------------------------------------------------------------------------------------------
\subsection{Categorias}
\label{categorias-ex}
\begin{titlemize}{Lista de dependências}
	\item \hyperref[categorias-def]{categorias-def};\\ %'dependencia1' é o label onde o conceito Dependência 1 aparece (--à arrumar um padrão para referencias e labels--) 

\end{titlemize}

\begin{ex}[Exemplos de Categorias]
	Alguns dos seguintes exemplos não serão tratados com detalhes. No entanto, pode-se consultá-los em quaisquer livros de teoria das categorias.
\begin{itemize}
\item \textbf{Mon} é uma categorial em que os objetos são monóides e os morfismos são homomorfismos de monóides.
\item \textbf{Grp} é a categoria dos grupos e homomorfismo de grupos (A categoria \textbf{Ab} é a categoria dos grupos abelianos.
\item A categoria \textbf{TOP} tem como objetos os espaços topológicos e como morfismos as funções contínuas (há também a categoria $\mathbf{TOP_*}$ dos espaços topológicos com um ponto selecionado, onde os morfismos $f:(X,x) \longrightarrow (Y,y)$ são funções contínuas tais que $f(x) = y$).
\item $\mathbf{Vec(\mathbb{K})}$ é a categoria dos espaços vetoriais sobre o corpo $\mathbb{K}$ e as transformações lineares dos espaços
\item A categoria \textbf{SET} tem como objetos os conjuntos e os morfismos são as funções entre os conjuntos. Ainda, pode-se definir $\mathbf{SET}_\omega$, a categoria dos conjuntos finitos e as funções entre eles.
\item A categoria \textbf{Ord} dos ordinais e das funções entre eles (funções entre esses conjuntos transitivos). Da mesma forma, $\mathbf{Ord}_\omega$ é a categoria dos ordinais finitos e as funções entre eles.

\item Uma relação $\leq$ é dita relação de ordem parcial se satisfaz:
\begin{itemize}
    \item $a \leq a$ para todo $a$.
    \item Se $a \leq b$ e $b \leq a$, então $a = b$ para todos $a$ e $b$.
    \item Se $a \leq b$ e $b \leq c$, então $a \leq c$ para todos $a$, $b$ e $c$.
\end{itemize}
A categoria $\mathbf{PO}$ (partial-order) é definida com morfismos estabelecendo a ordem entre os objetos, isto é, $A \leq B$ se, e somente se, existe $f$ em $Mor(\mathbf{PO})$, tal que % https://q.uiver.app/#q=WzAsMixbMCwwLCJBIl0sWzEsMCwiQiJdLFswLDEsImYiXV0=
\begin{tikzcd}[cramped]
	A & B
	\arrow["f", from=1-1, to=1-2]
\end{tikzcd}
\item Já a categoria \textbf{POS} tem como objetos os conjuntos parcialmente ordenados e os morfismos são funções que preservam a ordem, isto é, se $f:$ % https://q.uiver.app/#q=WzAsMixbMCwwLCJBIl0sWzEsMCwiQiJdLFswLDFdXQ==
\begin{tikzcd}[cramped]
	A & B
	\arrow[from=1-1, to=1-2]
\end{tikzcd}
, e $m \leq n$ em $A$, então $f(m) \leq f(n)$ em $B$.




\end{itemize}

\end{ex}


\begin{titlemize}{Lista de consequências}
	\item \hyperref[homotopia]{homotopia};\\ %'consequencia1' é o label onde o conceito Consequência 1 aparece
\end{titlemize}
