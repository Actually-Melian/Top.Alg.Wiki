%---------------------------------------------------------------------------------------------------------------------!Draft!-----------------------------------------------------------------------------------------------------------------
\subsection{Transformação Natural}
\label{transformação-natural-categorias-ex}
\begin{titlemize}{Lista de dependências}
	\item \hyperref[transformação-natural-categorias-def]{transformação-natural-categorias-def};\\ %'dependencia1' é o label onde o conceito Dependência 1 aparece (--à arrumar um padrão para referencias e labels--) 
	\item \hyperref[categorias-ex]{categorias-ex};\\
% quantas dependências forem necessárias.
\end{titlemize}

\begin{ex}[Transformações Naturais]
	Alguns exemplos de transformações naturais.
 
    \begin{itemize}
        \item $J:\mathbf{1_{\mathbf{Vec(\mathbb{K})}}} \Longrightarrow ()^{**} $ é uma transformação natural do funtor identidade no funtor bidual $()^{**}$, de tal forma que $J_X(x)(z^*) = z^*(x)$. Ainda, $f^{**}: A^{**} \longrightarrow B^{**}$ para algum morfismo $f: A \longrightarrow B$ é a transformação linear, tal que $f^{**}(z^{**})(x^*) = z^{**}(x^{*} \circ f)$. Então o seguinte diagrama comuta:
       % https://q.uiver.app/#q=WzAsNixbMiwwLCJWIl0sWzQsMCwiVl57Kip9Il0sWzIsMiwiVyJdLFs0LDIsIldeeyoqfSJdLFswLDAsIlYiXSxbMCwyLCJXIl0sWzQsNSwiTCIsMl0sWzAsMiwiTCIsMl0sWzEsMywiTF57Kip9IiwyXSxbMCwxLCJKX1YiLDEseyJzdHlsZSI6eyJ0YWlsIjp7Im5hbWUiOiJob29rIiwic2lkZSI6InRvcCJ9fX1dLFsyLDMsIkpfVyIsMSx7InN0eWxlIjp7InRhaWwiOnsibmFtZSI6Imhvb2siLCJzaWRlIjoidG9wIn19fV1d
\[\begin{tikzcd}
	V && V && {V^{**}} \\
	\\
	W && W && {W^{**}}
	\arrow["L"', from=1-1, to=3-1]
	\arrow["{J_V}"{description}, hook, from=1-3, to=1-5]
	\arrow["L"', from=1-3, to=3-3]
	\arrow["{L^{**}}"', from=1-5, to=3-5]
	\arrow["{J_W}"{description}, hook, from=3-3, to=3-5]
\end{tikzcd}\].

\item Temos o funtor $\#: \mathbf{SET_\omega} \longrightarrow \mathbf{Ord}_\omega$ que leva um conjuto finito em seu respectivo ordinal. Dessa forma, uma classe de bijeções $(\alpha_A: A \hookrightarrow \#(A))_{A \in Obj(\mathbf{SET_\omega})}$ define uma transformação natural.

% https://q.uiver.app/#q=WzAsNixbMiwwLCJBIl0sWzQsMCwiXFwjQSJdLFsyLDIsIkIiXSxbNCwyLCJcXCNCIl0sWzAsMCwiQSJdLFswLDIsIkIiXSxbNCw1LCJmIiwyXSxbMCwyLCJmIiwyXSxbMSwzLCJcXCNmIiwyXSxbMCwxLCJcXGFscGhhX0EiLDFdLFsyLDMsIlxcYWxwaGFfQiIsMV1d
\[\begin{tikzcd}
	A && A && {\#A} \\
	\\
	B && B && {\#B}
	\arrow["f"', from=1-1, to=3-1]
	\arrow["{\alpha_A}"{description}, from=1-3, to=1-5]
	\arrow["f"', from=1-3, to=3-3]
	\arrow["{\#f}"', from=1-5, to=3-5]
	\arrow["{\alpha_B}"{description}, from=3-3, to=3-5]
\end{tikzcd}\]


        
    \end{itemize}
\end{ex}


\begin{titlemize}{Lista de consequências}
	\item \hyperref[consequencia1]{Consequência 1};\\ %'consequencia1' é o label onde o conceito Consequência 1 aparece
	\item \hyperref[]{}
\end{titlemize}
