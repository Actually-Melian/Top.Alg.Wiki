%---------------------------------------------------------------------------------------------------------------------!Draft!-----------------------------------------------------------------------------------------------------------------
\subsection{Cone sobre subespaços de $\mathbb{R}^n$}
\label{cone-euclidiano-prop}
\begin{titlemize}{Lista de dependências}
	\item \hyperref[cone-def]{Cone sobre um Espaço Topológico}.
\end{titlemize}
Provaremos que a construção de cone sobre um subespaço topológico de $\mathbb{R}^n$ (com a topologia usual) é homeomorfa à construção geométrica em $\mathbb{R}^{n+1}$, sendo assim uma generalização de conceitos que antes só faziam sentido no contexto de espaços euclidianos.
\begin{thm}[A construção de cone generaliza a de subespaços euclidianos]
	Seja $X \subset \mathbb{R}^n$. Então,\begin{align*}
        \phi: C(X) &\rightarrow \{((1-t) x_1,...,(1-t) x_n,t): (x_1,...,x_n) \in X, t \in [0,1]\}\subset \mathbb{R}^{n+1}\\
        [(x,t)] \mapsto ((1-t) x_1,...,(1-t) x_n,t), \forall x=(x_1,...,x_n)\in X, \forall t \in [0,1]
    \end{align*}
    é um homeomorfismo.\\
    Ou seja, para subespaços de $\mathbb{R}^n$, a construção de cone recupera a intuição geométrica e coincide enquanto espaço topológico com $\{((1-t)x,t):x\in X, t\in I\} \in \mathbb{R}^{n+1}$, onde $(x,t)$ denota concatenação de vetores.
\end{thm}

Tal proposição é análoga à sobre \hyperref[suspensao-euclidiano-prop]{suspensão sobre um espaço euclidiano}.

\begin{titlemize}{Lista de consequências}
	%\item \hyperref[consequencia1]{Consequência 1}.
	%\item \hyperref[]{}
\end{titlemize}