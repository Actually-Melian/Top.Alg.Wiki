\subsection{Teorema do Ponto Fixo de Brower}
\label{teo-ponto-fixo-Brower}

\begin{titlemize}{Lista de Dependências}
    \item \hyperref[retração-def]{Dependência 1}
    \item \hyperref[grupo-fundamental]{Dependência 2}
\end{titlemize}]

\begin{lemma}
    Não existe uma retração $r:D^2 \longrightarrow \partial D^2 = S^1$
\end{lemma}

\begin{proof}
 Suponha que $r:D^2 \longrightarrow S^1$ seja uma retração. Sendo $D^2$ um espaço contrátil, pois ele é convexo, temos que para todo laço $\alpha: I \Longrightarrow D^2$ existe uma homotopia que leva esse laço no ponto $\alpha(0) = \alpha(1) = x_0$ de $D^2$. Em particular, para um laço $\beta: I \longrightarrow S^1 = \partial D^2$ em $S^1$ existe uma homotopia relativa a $\partial I$, $H: I\times I \longrightarrow D^2$ tal que $H(t, 0) = \beta(t)$ e $H(t, 1) = \beta(0) = x \in S^1$. Se a retração $r$ existe, então $r\circ H: I\times I: \longrightarrow S^1$ é uma homotopia relativa a $\partial I$. De fato, $(r\circ H)(0, t) = r(\beta(t)) = \beta(t)$ e $(r\circ H)(s, 0) = r(x) = x$ e $(r\circ H)(1, t) = r(x) = x$. Dessa forma, teríamos que $S^1$ é contrátil, contrariando $\pi_1(S^1) = \mathbb{Z}$.
\end{proof}

\begin{thm}
Toda função contínua na bola possui ponto fixo, i.e, se $f:D^2 \longrightarrow D^2$, então existe $x \in D^2$, tal que $f(x) = x$.
\end{thm} 

\begin{proof}
    Suponha por absurdo que exista uma função contínua $f:D^2 \longrightarrow D^2$ sem pontos fixos. Defino a função $\alpha: D^2 \longrightarrow S^1$ onde $\alpha(x)$ é o único ponto de intersecção da semirreta $\overrightarrow{f(x)x}$ com $S^1$. Essa função está bem definida, pois $t_x(\lambda) = \|f(x) + \lambda(x - f(x))\|$ é uma função real contínua tal que $t_x(0) \leq 1$ e $t_x \to \infty$ quando $x \to \infty$. Assim, pelo teorema do valor intermediário, existe $\lambda_x$ tal que $t_x(\lambda_x) = 1$. $\alpha$ é contínua, pois $\alpha(x) = f(x) + \lambda_x(x - f(x))$, onde $\lambda_x$ é expresso da seguinte forma:
    $\|f(x) + \lambda_x(x - f(x))\| = \|(x - f(x))\|^2\lambda_x^2 + 2\langle f(x), (x - f(x)) \rangle\lambda_x + \|f(x)\|^2 = 1$. Isso nos dá uma equação quadrática com 2 soluções reais, sendo a maior delas $\lambda_x = \frac{-2\langle f(x), x - f(x)\rangle + \sqrt{(2\langle f(x), x - f(x)\rangle)^2 - 4(\|x - f(x)\|^2)(\|f(x)\|^2 - 1)}}{2(\|x - f(x)\|}$. Note que não há problema com o quociente desde que assumimos por hipótese que $f(x) \ne x$ para todo $x$. Ainda, o termo dentro da raiz quadrada é sempre maior ou igual a 0, pela desigualdade de Schwarz. Além disso, se $x \in S^1$, então $\alpha(x) = x$. Portanto, $\alpha$ é uma retração, contrariando o lema anterior. 
\end{proof}