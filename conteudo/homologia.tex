\section{Homologia}
\label{homologia}

\begin{titlemize}{Lista de Dependências}
	\item \hyperref[homotopia]{Homotopia}.\\ %assunto1 é o label onde o Assunto 1 aparece
	%\item \hyperref[]{};
\end{titlemize}

Em topologia algébrica, a homologia é a sequência de grupos de homologia associada a um espaço topológico. Esses grupos capturam de forma algébrica a ideia dos "buracos" em diferentes dimensões no espaço. Assim, a homologia é uma ferramenta fundamental para distinguir e classificar espaços topológicos.

\subsection{Complexo de cadeias}
\label{complexo-de-cadeias-def}
\begin{titlemize}{Lista de dependências}
	\item %\hyperref[homologia-simplicial-def]{Homologia Simplicial};\\ %'dependencia1' é o label onde o conceito Dependência 1 aparece (--à arrumar um padrão para referencias e labels--) 
% quantas dependências forem necessárias.
\end{titlemize}
\begin{defi}[Complexo de cadeias]
	Um \textbf{Complexo de cadeias} é uma sequência $C_{-1}=0,C_0,C_1, C_2,...$ de grupos abelianos acompanhada de homomorfismos $d_n:C_n\rightarrow C_{n-1}$ para cada $n\ge 0$, tais que $d_{n}\circ d_{n+1}=0$. Denotamos esse complexo de cadeias por $C_{\bullet}$, e os homomorfismos $d_n$ são chamados de \textbf{diferenciais} de $C_\bullet$. A n-ésima \textbf{homologia} de $C_\bullet$ é definida por
    \[H_n(C_\bullet):=\frac{\text{Ker}(d_n)}{\text{Im}(d_{n+1})}\]
    Usualmente, $\text{Ker}(d_n)$ é chamado de grupo abeliano de $n$-ciclo em $C_\bullet$ e é denotado por $Z_n(C_\bullet)$. Por outro lado, $\text{Im}(d_{n+1})$ é chamada de grupo abeliano de $n$-bordos em $C_\bullet$ e é denotada por $B_n(C_\bullet)$.
\end{defi}

A homologia é a medida não numérica de quão diferentes $Z_n(C_\bullet)$ e $B_n(C_\bullet)$ são.

\begin{titlemize}{Lista de consequências}
	\item \hyperref[complexo-de-cadeias-def]{Complexo de cadeias};\\ 
    \item \hyperref[aplicacao-de-cadeias-def]{Aplicação de cadeias};\\
    \item \hyperref[homotopia-de-cadeias-def]{Homotopia de cadeia};\\
    \item \hyperref[homomorfismo-induzido-de-cadeias-prop]{Homomorfismo induzido de cadeias};\\
    \item \hyperref[equivalencia-de-homotopia-de-cadeias-def]{Equivalência de homotopia de cadeias};\\
    \item \hyperlink{homologia-simplicial-def}{Homologia simplicial};\\
    \item \hyperref[homologia-singular-def]{Homologia singular};\\
    \item \hyperref[homomorfismo-de-homologias-singulares-induzido-prop]{Homomorfismo de homologias singulares induzido}.
\end{titlemize}

\subsection{Aplicação de cadeias}
\label{aplicacao-de-cadeias-def}
\begin{titlemize}{Lista de dependências}
	\item \hyperref[complexo-de-cadeias-def]{Complexo de cadeias}.\\ %'dependencia1' é o label onde o conceito Dependência 1 aparece (--à arrumar um padrão para referencias e labels--) 
% quantas dependências forem necessárias.
\end{titlemize}

\begin{defi}
    Uma \textbf{aplicação de cadeias} ou homomorfismo de cadeias $f_\bullet:C_\bullet\rightarrow D_\bullet$ é uma sequência de homomorfismos $f_n:C_n\rightarrow D_n$ tal que $f_n\circ d_{n+1}^C=d_{n+1}^D\circ f_{n+1}$.
\end{defi}

\begin{titlemize}{Lista de consequências}
    \item \hyperref[homotopia-de-cadeias-def]{Homotopia de cadeia;}\\
    \item \hyperref[homomorfismo-induzido-de-cadeias-prop]{Homomorfismo induzido de cadeias};\\
    \item \hyperref[equivalencia-de-homotopia-de-cadeias-def]{Equivalência de homotopia de cadeias};\\
    \item \hyperref[homomorfismo-de-homologias-singulares-induzido-prop]{Homomorfismo de homologias singulares induzido}.
\end{titlemize}

\subsection{Homotopia de cadeias}
\label{homotopia-de-cadeias-def}
\begin{titlemize}{Lista de dependências}
	\item \hyperref[complexo-de-cadeias-def]{Complexo de cadeias};\\ %'dependencia1' é o label onde o conceito Dependência 1 aparece (--à arrumar um padrão para referencias e labels--) 
% quantas dependências forem necessárias.
    \item \hyperref[aplicacao-de-cadeias-def]{Aplicação de cadeias}.
\end{titlemize}

\begin{defi}
    Sejam $f_\bullet, g_\bullet:C_\bullet\rightarrow D_\bullet$ duas aplicações de cadeias. Uma \textbf{homotopia de cadeias} $h:f_\bullet\Rightarrow g_\bullet$ é uma sequência de homomorfismos $h_n:C_n\rightarrow D_{n+1}$, indexada por $n\ge -1$, tal que 
    \[g_n-f_n=d_{n+1}^D\circ h_n+ h_{n-1}\circ d_n^C:C_n\rightarrow D_n.\]
    Nessa caso, diremos que $f_\bullet$ e $g_\bullet$ são \textbf{homotópica de cadeias}, e denotaremos por $f_\bullet\simeq g_\bullet$.
\end{defi}

\begin{titlemize}{Lista de consequências}
    \item \hyperref[homomorfismo-induzido-de-cadeias-prop]{Homomorfismo induzido de cadeias};\\
    \item \hyperref[equivalencia-de-homotopia-de-cadeias-def]{Equivalência de homotopia de cadeias};\\
    \item \hyperref[homomorfismo-de-homologias-singulares-induzido-prop]{Homomorfismo de homologias singulares induzido}.\\
\end{titlemize}

\subsection{Homomorfismo induzido de cadeias} %afirmação aqui significa teorema/proposição/colorário/lema
\label{homomorfismo-induzido-de-cadeias-prop}
\begin{titlemize}{Lista de dependências}
	\item \hyperref[complexo-de-cadeias-def]{Complexo de cadeias};\\ 
    \item \hyperref[aplicacao-de-cadeias-def]{Aplicação de cadeias};\\
    \item \hyperref[homotopia-de-cadeias-def]{Homotopia de cadeia}.
\end{titlemize}
Assim como uma função contínua entre espaços topológicos induz um homomorfismo entre os grupos fundamentais associados, uma aplicação de cadeias induz um homomorfismo entre os grupos de homologias correspondentes.
\begin{lemma}%af(afirmação)/prop(proposição)/corol(corolário)/lemma(lema)/outros ambientes devem ser definidos no preambulo de Alg.Top-Wiki.tex 
	Uma aplicação de cadeias $f_\bullet: C_\bullet\rightarrow D_\bullet$ induz um homomorfismo 
    \begin{align*}
        f_*:H_n(C_\bullet)&\longrightarrow H_n(D_\bullet)\\
        [x]&\longmapsto [f_n(x)].
    \end{align*}
    Para todo $n\ge 0$. Além disso, se $f_\bullet$ e $g_\bullet$ são homotópicas de cadeias, então $f_*=g_*$.
\end{lemma}

\begin{proof}
    Vamos checar que $f_*$ é bem definido.
    \begin{itemize}
        \item Primeiramente, mostramos que $[f_n(x)]$ está dentro do codomínio. Seja $[x]\in H_n(C_\bullet)=\frac{Z_n(C_\bullet)}{B_n(C_\bullet)}$ representado por um $x\in Z_n(C_\bullet)$. Consideramos dois casos:\\
        Caso 1: $n=0$. Nesse caso, temos $Z_0(C_\bullet)=C_0$ e $Z_0(D_\bullet)=D_0$. Como $f_0(Z_0(C_\bullet))\subseteq Z_0(D_\bullet)$, temos que $f_0(x)$ é um ciclo, ou seja, $f_0(x)$ representa uma classe de homologia em $H_0(D_\bullet)$.\\
        Caso 2: $n\ge 1$. Como $x$ é um ciclo em $C_n$, temos:
        \[d_n^D\circ f_n(x)=f_{n-1}\circ d_n^C(x)=f_{n-1}(0)=0,\]
        ou seja, o elemento $f_n(x)\in D_n$ é um ciclo. Portanto $f_n(x)$ representa uma classe de homologia em $H_n(D_\bullet)$.
        \item Agora, suponha que $[x]=[y]$, ou seja $x-y\in B_n(C_\bullet)$. Logo, existe um $z\in C_{n+1}$ tal que $x-y=d_{n+1}^D(z)$. Então, temos: 
        \[f_n(x)-f_n(y)=f_n(d_{n+1}^C(z))=d_{n+1}^D (f_{n+1}(z))\]
        é um bordo. Portanto $[f_n(x)]=[f_n(y)]\in H_n(D_\bullet).$
    \end{itemize}
    Como $f_n$ são homomorfismos, o mapa $f_*$ também é um homomorfismo. 

    Agora, suponha que $h_\bullet: f_\bullet \Rightarrow g_\bullet$ é uma homotopia de cadeias. Seja $x\in Z_n(C_\bullet)$. Então, temos 
    \[g_n(x)-f_n(x)=d_{n+1}^D(h_n(x))+h_{n-1}(d_n^C(x)).\]
    Mas $x$ é um ciclo, logo $g_n(x)-f_n(x)=d_n(x)=0$. Isso mostra que $g_n(x)-f_n(x)$ é um bordo, portanto, $[g_n(x)]=[f_n(x)]$.
\end{proof}

De acordo com a construção do homomorfismo induzido, é fácil observar as seguintes propriedades.

\begin{corol}
    \begin{enumerate}
        \item Se $f_\bullet: C_\bullet\rightarrow D_\bullet$ e $g_\bullet:D_\bullet\rightarrow E_\bullet$ são aplicações de cadeias, então 
        \[(g_\bullet\circ f_\bullet)_*=g_*\circ f_*.\]
        \item $(id_{C_\bullet})_*=id_{H_n(C_\bullet)}$.
    \end{enumerate}

\end{corol}

\begin{titlemize}{Lista de consequências}
    \item \hyperref[equivalencia-de-homotopia-de-cadeias-def]{Equivalência de homotopia de cadeias};\\
    \item \hyperref[homomorfismo-de-homologias-singulares-induzido-prop]{Homomorfismo de homologias singulares induzido}.
	%\item \hyperref[consequencia1]{Consequência 1};\\ %'consequencia1' é o label onde o conceito Consequência 1 aparece
\end{titlemize}

\subsection{Equivalência de homotopia de cadeias} %afirmação aqui significa teorema/proposição/colorário/lema
\label{equivalencia-de-homotopia-de-cadeias-def}
\begin{titlemize}{Lista de dependências}
	\item \hyperref[complexo-de-cadeias-def]{Complexo de cadeias};\\ 
    \item \hyperref[aplicacao-de-cadeias-def]{Aplicação de cadeias};\\
    \item \hyperref[homotopia-de-cadeias-def]{Homotopia de cadeia};\\
    \item \hyperref[homomorfismo-induzido-de-cadeias-prop]{Homomorfismo induzido de cadeias}.
\end{titlemize}

\begin{defi}
    Uma aplicação de cadeias $f_\bullet:C\bullet\rightarrow D_\bullet$ é uma \textbf{equivalência de homotopia de cadeias} se existem uma aplicação de cadeias $g_\bullet:D_\bullet\rightarrow C_\bullet$ e homotopias de cadeias $f_\bullet\circ g_\bullet\simeq id_{D_\bullet}$ e $g_\bullet\circ f_\bullet \simeq id_{C_\bullet}$.
\end{defi}

Uma consequência imediata de \ref{homomorfismo-induzido-de-cadeias-prop} é: 

\begin{lemma}
    Se $f_\bullet:C_\bullet\rightarrow D_\bullet$ é uma equivalência de homotopia de cadeia, então $f_*:H_n(C_\bullet)\rightarrow H_n(D_\bullet)$ é um isomorfismo para todo $n\ge 0$.
\end{lemma}

\begin{proof}
    Por Lema \ref{homomorfismo-induzido-de-cadeias-prop}, temos
    \[f_*\circ g_*=(f_\bullet\circ g_\bullet)_*=(id_{D_\bullet})_*=id_{H_n(D_\bullet)},\]
    e vice-versa. 
\end{proof}

\begin{titlemize}{Lista de consequências}
    \item \hyperref[homomorfismo-de-homologias-singulares-induzido-prop]{Homomorfismo de homologias singulares induzido}.\\
	%\item \hyperref[]{}
\end{titlemize}

\subsection{Homologia Simplicial}
\label{homologia-simplicial-def}
\begin{titlemize}{Lista de dependências}
	\item \hyperref[complexo-de-cadeias-def]{Complexo de cadeias};\\ 
    \item \hyperref[aplicacao-de-cadeias-def]{Aplicação de cadeias};\\
    \item \hyperref[homotopia-de-cadeias-def]{Homotopia de cadeia}.
% quantas dependências forem necessárias.
\end{titlemize}
\begin{defi}
	Seja $K$ um complexo simplicial, e seja $O_n (K)$ um grupo abeliano livre com base dada por símbolos 
    \[\{[v_0,...,v_n]:v_0,v_1,...,v_n \text{ geram um simplexo em }K\}.\]
    Aqui, $v_i$ são considerado ordenado, e eles podem gerar um simplexo de dimensão menor que $n$ (i.e. a lista pode repetir).

    Seja $T_n(K)\le O_n(K)$ um subgrupo gerado por seguintes elementos 
    \begin{itemize}
        \item a sequência $[v_0,...,v_n]$ tem vertices repetidos,
        \item $[v_0,v_1,...,v_n]-sign(\sigma)\cdot[v_{\sigma(0)},v_{\sigma(1)},...,v_{\sigma(n)}]$, onde $\sigma$ é uma permutação em $\{0,1,...,n\}$. 
    \end{itemize}
    Definoms $C_n(K):=O_n(K)/T_n(K)$ como grupo quociente.
\end{defi}

\begin{defi}
    O \textbf{operador bordo} é um homomorfismo de grupo dado por 
    \begin{align*}
        d_n:C_n(K)&\longrightarrow C_{n-1}(K)\\
        [v_0,v_1,...,v_n]&\longmapsto \sum_{i=0}^n (-1)^i \cdot[v_0,v_1,...,\widehat{v_i},...,v_n],
    \end{align*}
    onde $[v_0,v_1,...,\widehat{v_i},...,v_n]$ denota a sequência obtida pela remoção de $v_i$.
\end{defi}

\begin{lemma}
    O homomorfismo $d_{n-1}\circ d_n:C_n(K)\rightarrow C_{n-2}(K)$ é nulo.
\end{lemma}

\begin{dem}
    Seja $[v_0,...,v_n]\in C_n(K)$, então 
    \begin{align*}
        d_{n-1}\circ d_n ( [v_0,...,v_n])&=d_{n-1}\Bigl(\sum_{i=0}^n (-1)^i \cdot[v_0,v_1,...,\widehat{v_i},...,v_n] \Bigr) \\
        &=\sum_{i=0}^n (-1)^i  \Bigl( \sum_{k=0}^{i-1}(-1)^k[v_0,...,\widehat{v_k},...,\widehat{v_i},...,v_n])\\
        &+\sum_{k=i}^{n-1} (-1)^k[v_0,...,\widehat{v_i},...,\widehat{v_{k+1}},...,v_n]  \Bigr).
    \end{align*}
    O coeficiente de $[v_0,...,\widehat{v_a},...,\widehat{v_b},...,v_n]$ é $(-1)^a(-1)^b$ de $k=a$ e $i=b$ mais $(-1)^a(-1)^{b-1}$ de $i=a$ e $k+1=b$. Assim, cada termo se cancela, o que implica que $d_{n-1}\circ d_n([v_0,...,v_n])=0$. Como $C_n(K)$ é gerado pelos simplexos $[v_0,...,v_n]$, concluímos que $d_{n-1}\circ d_n=0$.
\end{dem}

Como a consequência, esse lema garante que $\text{Im}(d_n)\subseteq \text{Ker}(d_{n-1})$. Ou seja, a sequência 
\[...\rightarrow C_{n+1}(K)\xrightarrow{d_{n+1}}C_n(K)\xrightarrow{d_n} C_{n-1}(K)\rightarrow...\rightarrow 0\]
é um complexo de cadeias.

\begin{defi}
    O n-ésima \textbf{grupo de homologia simplicial} de um complexo simplicial $K$ é 
    \[H_n(K):=\frac{\text{Ker}(d_n)}{\text{Im}(d_{n+1})}.\]
\end{defi}


\begin{titlemize}{Lista de consequências}
	\item %\hyperref[consequencia1]{Consequência 1};\\ %'consequencia1' é o label onde o conceito Consequência 1 aparece
	%\item \hyperref[]{}
\end{titlemize}

\subsection{Homologia-singular} %afirmação aqui significa teorema/proposição/colorário/lema
\label{homologia-singular-def}
\begin{titlemize}{Lista de dependências}
	\item \hyperref[complexo-de-cadeias-def]{Complexo de cadeias};\\ 
    \item \hyperref[aplicacao-de-cadeias-def]{Aplicação de cadeias};\\
    \item \hyperref[homotopia-de-cadeias-def]{Homotopia de cadeia}.\\
\end{titlemize}

\begin{defi}
    Seja $X$ um espaço topológico. Um p-\textbf{simplexo singular} em $X$ é uma função contínua 
    \[\phi:\Delta^p\longrightarrow X.\]
\end{defi}

\begin{defi}
    Se $\phi$ é um $p$-simplexo singular em um espaço topológico $X$, e $i$ é um inteiro tal que $0\le i\le p$, definimos $\partial_i (\phi)$, um (p-1)-simplexo singular em $X$, por 
    \[\partial_i \phi(t_0,...,t_{p-1})=\phi(t_0,...,t_{i-1},0,t_{i+1},...,t_{p-1}).\]
    Ou seja, $\partial_i \phi=\phi|_{[v_0,...,\widehat{v_i},...,v_{p}]}$ é a $i$-ésima face de $\phi$, obtida pela substituição do parâmetro $t_i$ por zero, onde $[v_0,...,v_p]=\Delta^p$
\end{defi}

\begin{defi}
    Seja $X$ um espaço topológico, definimos $S_n(X)$ como grupo abeliano livre cujo base é o conjunto de todos $n$-simplexos singulares de $X$. Um elemento de $S_n(X)$ é dito $n$-\textbf{cadeia singular} de $X$ e tem a forma 
    \[\sum_\phi n_\phi \phi\]
    onde $n_\phi$ é um inteiro igual a zero para todos, exceto um número finito de $\phi$.
\end{defi}

Podemos estender o operador de $i$-ésima face para um homomorfismo de $S_n(X)$ em $S_{n-1} (X)$. 

\begin{defi}
    Seja $X$ um espaço topológico, definimos o operador $\partial_i$ como
    \begin{align*}
        \partial_i: S_n(X)&\longrightarrow S_{n-1}(X)\\
        \sum_\phi n_\phi \phi&\longmapsto \sum_\phi n_\phi \partial_i\phi.
    \end{align*}
    O \textbf{operador bordo} é então um homomorfismo definido por
    \begin{align*}
        \partial_{(n)}=\sum_{i=0}^n (-1)^i \partial_i:S_n(X)\longrightarrow S_{n-1}(X).
    \end{align*}
    Para simplificar a notação, omitiremos o índice do operador $\partial_{(n)}$.
\end{defi}

\begin{lemma}
    O homomorfismo $\partial\circ \partial:S_n(X)\rightarrow S_{n-2}(X)$ é nulo.
\end{lemma}

\begin{dem}
    Seja $\phi\in S_n(X)$, então 
    \begin{align*}
        \partial\circ \partial(\phi)&=\partial\Bigl(\sum_{i=0}^n (-1)^i \partial_i\phi \Bigr) \\
        &=\sum_{i=0}^n (-1)^i  \Bigl( \sum_{k=0}^{i-1}(-1)^k \partial_k\circ\partial_i \phi)+\sum_{k=i}^{n-1} (-1)^k\partial_{k}\circ\partial_i \phi  \Bigr).
    \end{align*}
    Note que $\partial_k\circ\partial_i\phi=\partial_{i-1}\circ\partial_k \phi$ se $k<i$. Logo, o coeficiente de $\partial_a\circ\partial_b \phi$ é $(-1)^a(-1)^b$ de $k=a$ e $i=b$ mais $(-1)^a(-1)^{b-1}$ de $i=a$ e $k=b-1$. Assim, cada termo se cancela, o que implica que $\partial\circ\partial\phi=0$. Como $S_n(K)$ é gerado pelos n-simplexos singulares, concluímos que $\partial\circ\partial=0$.
\end{dem}

Como a consequência, esse lema garante que $\text{Im}(\partial_{(n+1)})\subseteq \text{Ker}(\partial_{(n)})$. Ou seja, a sequência 
\[...\rightarrow S_{n+1}(X)\xrightarrow{\partial}S_n(X)\xrightarrow{\partial} S_{n-1}(X)\rightarrow...\rightarrow 0\]
é um complexo de cadeias. Denotamos esse complexo por $S(X)_*$.

Assim como no complexo de cadeias, denotamos $\text{Im}(\partial_{(n+1)})$ por $B_n(X)$ e $\text{Ker}(\partial_{(n)})$ por $Z_n(X)$.

\begin{defi}
    O n-ésima \textbf{grupo de homologia singular} de um espaço topológico $X$ é 
    \[H_n(X):=\frac{Z_n(X)}{B_n(X)}.\]
\end{defi}

\begin{titlemize}{Lista de consequências}
    \item \hyperref[homomorfismo-de-homologias-singulares-induzido-prop]{Homomorfismo de homologias singulares induzido}.\\ %'consequencia1' é o label onde o conceito Consequência 1 aparece
	%\item \hyperref[]{}
\end{titlemize}

\subsection{Homomorfismo de homologias singulares induzido} %afirmação aqui significa teorema/proposição/colorário/lema
\label{homomorfismo-de-homologias-singulares-induzido-prop}
\begin{titlemize}{Lista de dependências}
	\item \hyperref[complexo-de-cadeias-def]{Complexo de cadeias};\\ 
    \item \hyperref[aplicacao-de-cadeias-def]{Aplicação de cadeias};\\
    \item \hyperref[homotopia-de-cadeias-def]{Homotopia de cadeia};\\
    \item \hyperref[homomorfismo-induzido-de-cadeias-prop]{Homomorfismo induzido de cadeias};\\
    \item \hyperref[equivalencia-de-homotopia-de-cadeias-def]{Equivalência de homotopia de cadeias};\\
    \item \hyperref[homologia-singular-def]{Homologia singular}.
\end{titlemize}

\begin{lemma}
    Uma função contínua $f:X\rightarrow Y$ entre espaços topológicos induz um homomorfismo de cadeia 
    \begin{align*}
    f_n:S_n(X)&\longrightarrow S_n(Y)\\
    \sum_{\phi}n_\phi\phi&\longmapsto \sum_\phi n_\phi (f\circ \phi).
    \end{align*}
\end{lemma}

\begin{dem}
    É fácil ver que $f_n$ é um homomorfismo de grupo, basta mostrar que $f_{n-1}\circ\partial=\partial\circ f_n$. Seja $\phi\in S_n(X)$. Como 
    \begin{align*}
        f_{n-1}\partial (\phi)=f_{n}(\sum_{i=0}^n (-1)^i \partial_i \phi)=\sum_{i=0}^n(-1)^i f\circ\partial_i\phi=\sum_{i=0}^n (-1)^i \partial_i(f\circ\phi)=\partial\circ f_n (\phi),
    \end{align*}
    podemos concluir que $f_\bullet:=(f_n)_{n\ge 0}$ é um homomorfismo de cadeias.
\end{dem}

Por lemma \ref{homomorfismo-induzido-de-cadeias-prop}, temos 

\begin{corol}
    Uma função contínua $f:X\rightarrow Y$ entre espaços topológicos induz um homomorfismo 
    \begin{align*}
        f_*: H_n(X)&\longrightarrow H_n(Y)\\
        [\sum_\phi n_\phi \phi]&\longmapsto [\sum_\phi n_\phi (f\circ \phi)]
    \end{align*} 
    entre homologias singulares.

    Além disso, temos 
    \begin{enumerate}
        \item $(f\circ g)_*=f_*\circ g_*$,
        \item $(id_X)_*=id_{H_n(X)}$ para todo $n\ge 0$.
    \end{enumerate}
\end{corol}
Isso mostra que a homologia singular é um invariante topológico.
\begin{corol}
    Se $f:X\rightarrow Y$ é um homeomorfismo, então $f_*:H_n(X)\rightarrow H_n(X)$ é um isomorfismo para todo $n\ge 0$. 
\end{corol}

\begin{proof}
    Seja $g:Y\rightarrow X$ a função inversa de $f$, pelo Corolário anterior, temos 
    \[f_*\circ g_*=(f\circ g)_*=(id_Y)_*=id_{H_n(Y)},\]
    e vice-versa.
\end{proof}

\begin{thm}
    Sejam $f,g:X\rightarrow Y$ funções contínuas entre espaços topológicos. Se $F:X\times I\rightarrow Y$ é uma homotopia de $f$ em $g$. Então, $f$ e $g$ induzem um mesmo homomorfismo $f_*=g_*:H_n(X)\rightarrow H_n(Y).$
\end{thm}

\begin{dem}
    A demonstração é baseada no Algebraic Topology do Allen Hatcher; o leitor pode encontrar uma interpretação geométrica dessa prova no livro.

    O ponto crucial é um procedimento para subdividir $\Delta^n\times I$ em simplexos. Em $\Delta^n\times I$, seja $\Delta^n\times \{0\}=[v_0,...,v_n]$ e $\Delta^n\times\{1\}=[w_0,...,w_n]$, onde $v_i$ e $w_i$ possuem a mesma imagem sob a projeção $\Delta^n\times I\rightarrow \Delta^n$. Podemos passar de $[v_0,...,v_n]$ para $[w_0,...,w_n]$ interpolando uma sequência de $n$-simplexos, cada um obtido do anterior movendo um vértice $v_i$ até $w_i$, começando com $v_n$ e trabalhando para trás até $v_0$. Portanto, o primeiro passo é mover $[v_0,...,v_n]$ para cima até $[v_0,...,v_{n-1},w_n],$ então o segundo passo é mover isso para $[v_0,...,v_{n-2}, w_{n-1},w_n]$ e assim por diante. Na etapa típica $[v_0,...,v_{i},w_{i+1},...,w_n]$ move-se para cima até $[v_0,...,v_{i-1},w_i,...,w_n]$. A região entre esses dois simplexos é exatamente o (n+1)-simplexo $[v_0,...,v_i,w_i,...,w_n]$ que tem $[v_0,...,v_i,w_{i+1},...,w_n]$ como face inferior e $[v_0,...,v_{i-1},w_i,...,w_n]$ como face superior. Em conjunto, $\Delta^n\times I$ é a união de (n+1)-simplexos $[v_0,...,v_i,w_i,...,w_n]$, cada um intersectando o próximo em uma face.

    Agora, definimos \textbf{operador prisma} $P:S_n(X)\rightarrow S_{n+1}(Y)$ pela seguinte fórmula 
    \[P(\phi)=\sum_{i=0}^n (-1)^i F\circ (\phi\times id_I)|_{[v_0,...,v_i,w_i,...,w_n]},\]
    onde $\phi$ é um $n$-simplexo singular. Vamos mostrar que esses operadores prisma satisfazem a seguinte relação 
    \[\partial P=g_\bullet-f_\bullet-P\partial.\]
    Geometricamente, o lado esquerdo da equação representa o bordo da prisma, e os três termos do lado direito representam a base superior $\Delta^n\times \{1\}$, a base inferior $\Delta^n\times\{0\}$, e os lados $\partial \Delta^n\times I$ da prisma. Para provar a relação, calculamos 
    \begin{align*}
        \partial P(\phi)=&\sum_{i=0}^{n} \Bigl(\sum_{j=0}^{i} (-1)^i(-1)^j F\circ (\phi\times id_I)|_{[v_0,...,\widehat{v_j},...,v_i, w_i,...,w_n]} \\
        &+ \sum_{j=i}^{n} (-1)^i(-1)^{j+1} F\circ (\phi\times id_I)|_{[v_0,...,v_i,w_i,...,\widehat{w_j},...,w_n]}\Bigr)
    \end{align*}
    Ou seja, 
    \begin{align*}
        \partial P(\phi)=&\sum_{j\le i\le n} (-1)^i(-1)^j F\circ (\phi\times id_I)|_{[v_0,...,\widehat{v_j},...,v_i, w_i,...,w_n]} \\
        &+ \sum_{i\le j\le n} (-1)^i(-1)^{j+1} F\circ (\phi\times id_I)|_{[v_0,...,v_i,w_i,...,\widehat{w_j},...,w_n]}
    \end{align*}
    Os termos com $i=j$ nas duas somas se cancelam exceto para $F\circ(\phi\times id_I)|_{[\widehat{v_0},w_0,...,w_n]}$, que é $g\circ\phi=g_\bullet (\phi)$, e $-F\circ (\phi\times id_I)|_{[v_0,...,v_n,\widehat{w_n}]},$ que é $-f\circ\phi=-f_\bullet(\phi)$. Os termos com $i\ne j$ são exatamente $-P\partial (\phi)$, pois 
    \begin{align*}
        P\partial(\phi)=&\sum_{j=0}^{n} \Bigl( \sum_{i=j+1}^{n} (-1)^{i-1}(-1)^j F\circ (\phi\times id_I)|_{[v_0,...,\widehat{v_j},...,v_i,w_i,...,w_n]}\\
        &+\sum_{i=0}^{j-1} (-1)^i(-1)^j F\circ(\phi\times id_I)|_{[v_0,...,v_i,w_i,...,\widehat{w_j},...,w_n]}\Bigr)
    \end{align*}
    ou seja,
    \begin{align*}
        P\partial(\phi)=&\sum_{j<i\le n} (-1)^{i-1}(-1)^j F\circ (\phi\times id_I)|_{[v_0,...,\widehat{v_j},...,v_i,w_i,...,w_n]}\\
        &+\sum_{i<j\le n} (-1)^i(-1)^j F\circ(\phi\times id_I)|_{[v_0,...,v_i,w_i,...,\widehat{w_j},...,w_n]}\Bigr).
    \end{align*}
    Portanto, $P$ é uma homotopia de cadeias de $f$ e $g$. Pelo Lema \ref{homomorfismo-induzido-de-cadeias-prop}, temos que $f_*=g_*$.
\end{dem}

\begin{corol}
    Se dois espaços topológicos $X,Y$ são equivalentes homotópicos, então $H_n(X)\cong H_n (Y)$ para todo $n\ge 0$. 
\end{corol}

\begin{titlemize}{Lista de consequências}
	\item \hyperref[consequencia1]{Consequência 1};\\ %'consequencia1' é o label onde o conceito Consequência 1 aparece
\end{titlemize}

