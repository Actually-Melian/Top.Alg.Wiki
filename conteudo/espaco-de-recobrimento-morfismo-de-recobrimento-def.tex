\subsection{Morfismo de recobrimento}
\label{morfismo-de-recobrimento-def}
\begin{titlemize}{Lista de dependências}
	\item \hyperref[espaco-de-recobrimento-def]{Espaço de recobrimento};\\
	\item \hyperref[levantamento-de-funções-prop]{Levantamento de funções};
\end{titlemize}
\begin{defi}[Morfismo de recobrimento]
	Um morfismo de recobrimento entre dois recobrimentos $p_1:E_1\longrightarrow X$ e $p_2:E_2\longrightarrow X$ é uma aplicação contínua $\phi:E_1\longrightarrow E_2$ tal que $p_2\circ\phi = p_1$.
\end{defi}

\[\begin{tikzcd}
	{E_1} && {E_2} \\
	\\
	& X
	\arrow["\phi"', from=1-1, to=1-3]
	\arrow["{p_1}", from=1-1, to=3-2]
	\arrow["{p_2}"', from=1-3, to=3-2]
\end{tikzcd}\]

Note que $\phi$ é um levantamento de $p_1$ para o recobrimento $p_2:E_2\longrightarrow X$.

\[\begin{tikzcd}
	&& {E_2} \\
	\\
	{E_1} && X
	\arrow["{p_2}"', from=1-3, to=3-3]
	\arrow["\phi"', from=3-1, to=1-3]
	\arrow["{p_1}", from=3-1, to=3-3]
\end{tikzcd}\]

Logo, assumindo que $E_1$ é conexo por caminhos, temos que se $\phi_1$ e $\phi_2$ são morfismos de recobrimento, então $\phi_1 = \phi_2 \Longleftrightarrow \phi(e_o) = \phi(e_0)$. Além disso, $p_{1*}(\pi_1(E_1,e_o))$ é um subgrupo de $p_{2*}(\pi_1(E_2,e_o))$.

\begin{titlemize}{Lista de consequências}
	\item \hyperref[recobrimento-universal]{Recobrimento universal};
\end{titlemize}
