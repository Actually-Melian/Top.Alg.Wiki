%---------------------------------------------------------------------------------------------------------------------!Draft!-----------------------------------------------------------------------------------------------------------------
\subsection{Cone sobre um Espaço Topológico}
\label{cone-def}
\begin{titlemize}{Lista de dependências}
	\item \hyperref[topologia-quociente-def]{Topologia Quociente}.
\end{titlemize}
\begin{def}[Cone]
	Dado um espaço topológico $X$, definimos o \bf{cone sobre X} como o espaço quociente $X\times I/\sim$, onde $I=[0,1]$ é o intervalo com a topologia usual de subespaço de $\mathbb{R}$, e para todos $(x,s),(y,t) \in X\times I$,\[
    (x,s)\sim(y,t) \\Leftrightarrow (x,s)=(y,t)\text{ ou }s=t=1.
    \]
\end{def}

É simples ver que a relação definida acima é de equivalência.

Tal construção é semelhante a de \item \hyperref[suspensao-def]{suspensão sobre um espaço topológico}.

\begin{titlemize}{Lista de consequências}
	%\item \hyperref[consequencia1]{Consequência 1};\\ %'consequencia1' é o label onde o conceito Consequência 1 aparece
	%\item \hyperref[]{}
\end{titlemize}