%-------------------------------------------------------------------------------------------------------------!Draft!-------------------------------------------------------------------------------------------------------------------------
\section{Topologia quociente}
\label{topologia-quociente}
Um assunto que aparece de forma recorrente na topologia algébrica é o conceito de topologia quociente, que exploraremos a seguir. 

\subsection{Topologia Quociente}
\label{topologia-quociente-def}
\begin{titlemize}{Lista de dependências}
	\item \hyperref[topologia final]{Topologia Final}; 
\end{titlemize}
\begin{defi}[Topologia Quociente]
	Seja \(X\) um espaço topológico e \(\sim\) uma relação de equivalência em \(X\).
	Podemos conferir ao espaço \(X/\sim\) uma estrutura de espaço topologíco da seguinte maneira. Considere a função projeção
	\begin{align*}
		\pi:X&\to X/\sim;\\
		x&\mapsto [x].
	\end{align*}
	Podemos fazer com que \(\pi\) seja uma função contínua dando, para \(X/\sim\) a topologia final com relação à \(\pi\). Isto é, os abertos de \(X/\sim\) são exatamente imagens de abertos em \(X\) por \(\pi\).  
\end{defi}
Varios exemplos importantes de espaços topológicos com os quais trabalharemos no estudo de topologia algébrica podem ser construídos como espaços quocientes. Em particular uma construção muito útil é a de quocientar um espaço por um subespaço, como explicado na seguinte definição.
\begin{defi}[Quociente por um subespaço]
	Seja \(X\) um espaço topológico e \(A \subseteq X\) um subespaço. Definimos a seguinte relação binária, \(\sim_{A}\):\\
	\(a\sim b\) se e somente se \(a=b\) ou \(a,b\in A\). Essa relação é uma relação de equivalência, e assim define um espaço \(X/\sim_A\). Esse espaço será denotado por \(X/A\). 
\end{defi}
Um exemplo desse tipo de construção é a esfera \(S^1\) que pode ser construida como \(I/\{0,1\}\) onde \(I=[0,1]\). 
\begin{titlemize}{Lista de consequências}
	\item \hyperref[pinched-torus-ex]{Torus Pinçado};
\end{titlemize}


% onde conteudos.tex é o nome do arquivo tex que voce quer incluir nessa secção.
\subsection{Função contínua em topologia quociente}
\label{funcao-continua-em-topologia-quociente-prop}
\begin{titlemize}{Lista de dependências}
	\item \hyperref[topologia-quociente-def]{Topologia quociente}; 
\end{titlemize}

\begin{prop}
    Sejam $X,Y$ espaços topológicos. E seja $\sim$ uma relação de equivalência em $X$. Uma função $f:(X/\sim) \longrightarrow Y$ é contínua se e somente se $f\circ \pi$ é contínua, onde $\pi$ é a função projeção associada ao quociente.
\end{prop}
\begin{dem}
    Por um lado, suponhamos que $f$ seja contínua. Como a composição de funções contínuas é contínua, a função $f\circ \pi$ também seré contínua.

    Por outro lado, suponhamos que $f\circ \pi$ seja contínua. Seja $V\subseteq Y$ um aberto, pela hipóteses, $\pi^{-1}(f^{-1}(V))$ é um aberto. Agora, pela definição de topologia quociente, $f^{-1}(V)$ é um aberto em $X/\sim$, o que implica que $f$ é contínua. 
\end{dem}

Dois exemplos importantes de espaços quociente são os seguintes.
%-------------------------------------------------------------------------------------------------------------!Draft!-------------------------------------------------------------------------------------------------------------------------
\section{Cone e Suspensão sobre Espaços Topológicos}
\label{cone-suspensao}
%---------------------------------------------------------------------------------------------------------------------!Draft!-----------------------------------------------------------------------------------------------------------------
\subsection{Cone sobre um Espaço Topológico}
\label{cone-def}
\begin{titlemize}{Lista de dependências}
	\item \hyperref[topologia-quociente-def]{Topologia Quociente}.
\end{titlemize}

\begin{defi}[Cone]
    Dado um espaço topológico $X$, definimos o \textbf{cone sobre $X$} como o espaço quociente $X\times I/\sim$,
    onde $I=[0,1]$ é o intervalo com a topologia usual de subespaço de $\mathbb{R}$, e $\sim$ é a relação de equivalência tal que para todos $(x,s),(y,t) \in X\times I$,\\
    \centerline{
    $(x,s)\sim(y,t) \Leftrightarrow (x,s)=(y,t)$ ou $s=t=1$.}
\end{defi}

Tal construção é semelhante a de \hyperref[suspensao-def]{suspensão sobre um espaço topológico}.

\begin{titlemize}{Lista de consequências}
	%\item \hyperref[consequencia1]{Consequência 1};\\ %'consequencia1' é o label onde o conceito Consequência 1 aparece
	%\item \hyperref[]{}
    \item \hyperref[suspensao-cone-duplo-prop]{A construção de suspensão coincide com a de cone duplo}
\end{titlemize}
%---------------------------------------------------------------------------------------------------------------------!Draft!-----------------------------------------------------------------------------------------------------------------
\subsection{Suspensão sobre um Espaço Topológico}
\label{suspensao-def}
\begin{titlemize}{Lista de dependências}
	\item \hyperref[topologia-quociente-def]{Topologia Quociente}.
\end{titlemize}
\begin{defi}[Suspensão]
	Dado um espaço topológico $X$, definimos a \textbf{suspensão sobre X} como o espaço quociente $X\times [-1,1]/\sim$, onde $[-1,1]$ é munido da topologia usual de subespaço de $\mathbb{R}$, e $\sim$ é a relação de equivalência tal que para todos $(x,s),(y,t) \in X\times [-1,1]$,\\
    \centerline{
    $(x,s)\sim(y,t) \Leftrightarrow (x,s)=(y,t)$ ou $s=t=1$ ou $s=t=-1$.}
\end{defi}

Tal construção é semelhante a de \hyperref[cone-def]{cone sobre um espaço topológico}.

\begin{titlemize}{Lista de consequências}
	%\item \hyperref[consequencia1]{Consequência 1};\\ %'consequencia1' é o label onde o conceito Consequência 1 aparece
	\item \hyperref[suspensao-cone-duplo-prop]{A construção de suspensão coincide com a de cone duplo}
\end{titlemize}
Seguem alguns resultados relevantes sobre cones e suspensões:
%---------------------------------------------------------------------------------------------------------------------!Draft!-----------------------------------------------------------------------------------------------------------------
\subsection{Suspensão coincide com Cone Duplo}
\label{suspensao-cone-duplo-prop}
\begin{titlemize}{Lista de dependências}
	\item \hyperref[cone-def]{Cone};\\
	\item \hyperref[suspensao-def]{Suspensão}.
\end{titlemize}
Provaremos que a suspensão sobre um espaço topológico é homeomorfa a um cone duplo, confirmando as intuições sobre os objetos.
\begin{prop}[A construção de suspensão coincide com a de cone duplo]
	Dado $X$ um espaço topológico, considere os espaços
    \[S(X) = X\times[-1,1]/\sim_S\hspace{6 pt} = \{[x,t]_S : x\in X, t\in [-1,1]\},\]
    \[C(X) = X\times I/\sim_C\hspace{6 pt} = \{[x,t]_C : x\in X, t\in I\}.\]
    \\
    \noindent E sobre $C(X) \amalg C(X) = C(X)\times\{-1,1\}$, seja $\sim$ a relação de equivalência\\
    \\ \centerline{
    $([x,s]_C,i) \sim ([y,t]_C,j) \Leftrightarrow ([x,s]_C,i) = ([y,t]_C,j)$ ou $(x=y$ e $s=t=0)$}\\
    \\para todos $x,y \in X$, $s,t \in I$ e $i,j \in \{-1,1\}$.    
    
    Defina a função \begin{align*}
        \psi: S(X) &\rightarrow (C(X) \amalg C(X))/ \sim\\
                [x,t]_S &\mapsto [[x,|t|]_C,\text{sgn}(t)]
    \end{align*}
    onde $\text{sgn}(t) = -1$ caso $t < 0$ e $\text{sgn}(t) = 1$ caso contrário. Então $\psi$ é homeomorfismo.

	Ou seja, a construção de suspensão sobre um espaço topológico coincide com a colagem de dois cones sobre o mesmo espaço topológico, quando identificamos as suas respectivas bases.

    \begin{dem}
        Primeiramente, $\psi$ está bem definida. Dado $(x,t) \in X\times I$, vale que $[x,t]_S = \{(x,t)\}$, se $t \in \left]-1,1\right[$, ou então $[(x,t)] = X\times\{t\}$, se $t \in \{-1,1\}$. Mas para quaisquer $x,x'\in X$ e $t\in \{-1,1\}$,\[
        [[x,|t|]_C,\text{sgn}(t)] =
        [X\times\{|t|\},\text{sgn}(t)] =
        [[x',|t|]_C,\text{sgn}(t)]
        \]
        e então está bem definido $\psi([x,t]_S)$.

        Denotemos por $\pi_S$, $\pi_C$ e $\pi$ as aplicações quociente canônicas com respeito a $\sim_S$, $\sim_C$ e $\sim$, respectivamente. Por \textbf{(adicionar proposição aqui)}, $\psi$ é contínua se, e somente se,\begin{align*}
            \psi \circ \pi_S: X\times[-1,1] &\rightarrow (C(X) \amalg C(X))/ \sim\\
            (x,t) &\mapsto [[x,|t|]_C,\text{sgn}(t)]
        \end{align*}
        for contínua. Para todo subconjunto aberto $U \subset C(X)\amalg C(X)/\sim\hspace{3pt}$, existem abertos $V,W \subset C(X)$ tais que $\pi^{-1}(U) = V\amalg W = (V\times\{-1\}) \cup (W\times\{1\})$. Por sua vez, $\pi_C^{-1}(V)$ e $\pi_C^{-1}(W)$ são abertos de $X\times[0,1]$. Então,
        \begin{align*}
        (\psi \circ \pi_S)^{-1}(U)
        &= (\psi \circ \pi_S)^{-1}(\pi(V\times\{-1\}) \cup \pi(W\times\{1\}))\\
        &= (\psi \circ \pi_S)^{-1}(\pi(V\times\{-1\})) \cup (\psi \circ \pi_S)^{-1}(\pi(W\times\{1\})).   
        \end{align*}
        
        Calculemos cada termo da união separadamente.
        \begin{align*}
            (\psi \circ \pi_S)^{-1}(\pi(V\times\{-1\}))
            &= \{(x,t): [x,|t|]_C \in V, t \in [-1,0]\}\\
            &= \{(x,-t): [x,t]_C \in V, t \in I\}\\
            &= A(\pi_C^{-1}(V))   
        \end{align*}
        onde $A: (x,t) \mapsto (x,-t)$. E analogamente, $(\psi \circ \pi_S)^{-1}(\pi(W\times\{1\})) = \pi_C^{-1}(W)$. Concluímos assim que
        \[(\psi \circ \pi_S)^{-1}(U) = A(\pi_C^{-1}(V)) \cup \pi_C^{-1}(W)\]
        é aberto, e então $\psi$ é contínua.
    \end{dem}
\end{prop}


\begin{titlemize}{Lista de consequências}
	%\item \hyperref[consequencia1]{Consequência 1};\\ %'consequencia1' é o label onde o conceito Consequência 1 aparece
	%\item \hyperref[]{}
    \item \hyperref[suspensao-euclidiano-prop]{Suspensão sobre subespaços de $\mathbb{R}^n$}
    \item \hyperref[suspensao-esfera-prop]{Suspensão sobre esferas $S^n\subset\mathbb{R}^{n+1}$}
\end{titlemize}


%---------------------------------------------------------------------------------------------------------------------!Draft!-----------------------------------------------------------------------------------------------------------------
\subsection{Cone sobre subespaços de $\mathbb{R}^n$}
\label{cone-euclidiano-prop}
\begin{titlemize}{Lista de dependências}
	\item \hyperref[cone-def]{Cone sobre um Espaço Topológico}.
\end{titlemize}
Provaremos que a construção de cone sobre um subespaço topológico de $\mathbb{R}^n$ (com a topologia usual) é homeomorfa à construção geométrica em $\mathbb{R}^{n+1}$, sendo assim uma generalização de conceitos que antes só faziam sentido no contexto de espaços euclidianos.
\begin{thm}[A construção de cone generaliza a de subespaços euclidianos]
	Seja $X \subset \mathbb{R}^n$. Então,\begin{align*}
        \phi: C(X) &\rightarrow \{((1-t) x_1,...,(1-t) x_n,t): (x_1,...,x_n) \in X, t \in [0,1]\}\subset \mathbb{R}^{n+1}\\
        [(x,t)] \mapsto ((1-t) x_1,...,(1-t) x_n,t), \forall x=(x_1,...,x_n)\in X, \forall t \in [0,1]
    \end{align*}
    é um homeomorfismo.\\
    Ou seja, para subespaços de $\mathbb{R}^n$, a construção de cone recupera a intuição geométrica e coincide enquanto espaço topológico com $\{((1-t)x,t):x\in X, t\in I\} \in \mathbb{R}^{n+1}$, onde $(x,t)$ denota concatenação de vetores.
\end{thm}

Tal proposição é análoga à sobre \hyperref[suspensao-euclidiano-prop]{suspensão sobre um espaço euclidiano}.

\begin{titlemize}{Lista de consequências}
	%\item \hyperref[consequencia1]{Consequência 1}.
	%\item \hyperref[]{}
\end{titlemize}
%---------------------------------------------------------------------------------------------------------------------!Draft!-----------------------------------------------------------------------------------------------------------------
\subsection{Suspensão sobre subespaços de $\mathbb{R}^n$}
\label{suspensao-euclidiano-prop}
\begin{titlemize}{Lista de dependências}
	\item \hyperref[suspensao-def]{Suspensão sobre um Espaço Topológico};
    \item \hyperref[suspensao-cone-duplo-prop]{Suspensão e Cone Duplo}
    \item \hyperref[cone-euclidiano-prop]{Cone sobre subespaços de $\mathbb{R}^n$}
\end{titlemize}
Provaremos que a construção de uma suspensão sobre um subespaço topológico de $\mathbb{R}^n$ (com a topologia usual) é homeomorfa à construção geométrica em $\mathbb{R}^{n+1}$, sendo assim uma generalização de conceitos que antes só faziam sentido no contexto de espaços euclidianos.

Nesse momento, dados $(x_1,x_2,...,x_n) \in \mathbb{R}^n$ e $t \in \mathbb{R}$, denotemos por $(x,t)$ o vetor $(x_1,x_2,...,x_n,t) \in \mathbb{R}^{n+1}$.

\begin{prop}[A construção de suspensão generaliza a de subespaços euclidianos]
	Seja $X \subset \mathbb{R}^n$, e considere\\
    \centerline{$S_g(X) = \{((1-|t|)x,t):x\in X, t\in [-1,1]\} \subset \mathbb{R}^{n+1}$.}\\
    Então,\begin{align*}
        \varphi: S(X) &\rightarrow S_g(X)\\
        [(x,t)] &\mapsto ((1-t)x,t), \forall x\in X, \forall t \in [-1,1]
    \end{align*}
    é um homeomorfismo.\\
    Ou seja, para subespaços de $\mathbb{R}^n$, a construção de suspensão recupera a intuição geométrica e coincide enquanto espaço topológico com $S_g(X)$.
\end{prop}

Tal proposição é análoga à sobre \hyperref[cone-euclidiano-prop]{cone sobre um espaço euclidiano}.

%\begin{titlemize}{Lista de consequências}
	%\item \hyperref[consequencia1]{Consequência 1};\\ %'consequencia1' é o label onde o conceito Consequência 1 aparece
	%\item \hyperref[]{}
%\end{titlemize}

%---------------------------------------------------------------------------------------------------------------------!Draft!-----------------------------------------------------------------------------------------------------------------
\subsection{Cone sobre esferas $S^n\subset\mathbb{R}^{n+1}$}
\label{cone-esfera-prop}
\begin{titlemize}{Lista de dependências}
	\item \hyperref[cone-def]{Cone sobre um Espaço Topológico}.
    %Definição de esfera e disco
\end{titlemize}

Nesse momento, dados $(x_1,x_2,...,x_n) \in \mathbb{R}^n$ e $t \in \mathbb{R}$, denotemos por $(x,t)$ o vetor $(x_1,x_2,...,x_n,t) \in \mathbb{R}^{n+1}$.

\begin{prop}[Cone sobre esferas]
	$C(S^n) \cong D^{n+1}$, para todo $n\geq 1$.
 
    \begin{dem}
        Pela \hyperref[cone-euclidiano-prop]{Proposição acerca de cones sobre subconjuntos de $\mathbb{R}^n$}, basta provar que $D^{n+1}\cong C_g(S^n) = \{((1-t)x,t):x\in S^n, t\in I\}$.

        Seja $\pi:\mathbb{R}^{n+1}\to\mathbb{R}^n$ a projeção dada por $\pi(x,t) = x$ para cada $(x,t) \in \mathbb{R}^{n+1}$.
        
        Defina $f:D^{n+1}\to C_g(S^n)$ como $f(p) = (p,1-\|p\|)$. $f$ está bem definida, pois $(p,1-\|p\|)=((1-t)x,t)$ para $t=1-\|p\|$, e $x=p/\|p\|$ caso $\|p\|\neq 0$ ou então $x\in S^n$ qualquer caso contrário. Tal raciocínio também mostra que $f$ é sobrejetora. $f$ é injetora, uma vez que $\pi \circ f(p) = p$.

        É fácil ver que $f$ é contínua.  Pelo Teorema de Heine-Borel, $D^{n+1} \subset \mathbb{R}^{n+1}$ é compacto, assim $f(D^{n+1})=C_g(S^n)$ também é. E ambos são Hausdorff, portanto $f$ é homeomorfismo e concluímos. 
    \end{dem}
\end{prop}

Tal proposição é análoga à sobre \hyperref[suspensao-esfera-prop]{suspensão sobre esferas}.

\begin{titlemize}{Lista de consequências}
	%\item \hyperref[consequencia1]{Consequência 1}.
	%\item \hyperref[]{}
\end{titlemize}
%---------------------------------------------------------------------------------------------------------------------!Draft!-----------------------------------------------------------------------------------------------------------------
\subsection{Suspensão sobre esferas $S^n\subset\mathbb{R}^{n+1}$}
\label{suspensao-esfera-prop}
\begin{titlemize}{Lista de dependências}
	\item \hyperref[suspensao-def]{Suspensão sobre um Espaço Topológico}
    \item \hyperref[suspensao-cone-duplo-prop]{Suspensão coincide com Cone Duplo}
    \item \hyperref[cone-esfera-prop]{Cone sobre esferas $S^n\subset\mathbb{R}^{n+1}$}
    %Definição de esfera e disco
\end{titlemize}

%Nesse momento, dados $(x_1,x_2,...,x_n) \in \mathbb{R}^n$ e $t \in \mathbb{R}$, denotemos por $(x,t)$ o vetor $(x_1,x_2,...,x_n,t) \in \mathbb{R}^{n+1}$.

Tal proposição é análoga à sobre \hyperref[cone-esfera-prop]{cone sobre esferas} e poderia ser provada de maneira análoga, porém a utilizaremos para facilitar a demonstração.

\begin{prop}[Suspensão sobre esferas]
	$S(S^n) \cong S^{n+1}$, para todo $n\geq 1$.

    \begin{dem}
        Pela \hyperref[suspensao-cone-duplo-prop]{caracterização de suspensão como cone duplo}, \[S(S^n) \cong (C(S^n) \amalg C(S^n))/\sim_0,\] onde $\sim_0$ identifica as bases dos cones. Já pela \hyperref[cone-esfera-prop]{Proposição acerca de cones sobre esferas}, $C(S^n) \cong D^{n+1}$, logo \[S(S^n) \cong (D^{n+1} \amalg D^{n+1})/\sim\] onde $\sim$ identifica os subconjuntos $S^n$ contidos em cada $D^{n+1}$.
        
        Basta então provar que o disco $D^{n+1}$ é homeomorfo a um hemisfério $H^{n+1} = S^{n+1}\cap (\mathbb{R}^n \times [0,\infty[)$. Seja $\pi:\mathbb{R}^{n+1}\to\mathbb{R}^n$ a projeção dada por $\pi(x,t) = x$ para cada $(x,t) \in \mathbb{R}^{n+1}$.
        
        De fato, $h:D^{n+1}\to H^{n+1}$ dada por $h(p)=(p,\sqrt{1-\|p\|^2})$ está bem definida, é contínua e é injetora pois $\pi \circ h(p) = p$. Por fim, é sobrejetora pois para todo ponto $(p,t)\in H^{n+1}$ vale que $t=\sqrt{1-\|p\|^2}$. Portanto, pelo mesmo argumento que na \hyperref[cone-esfera-prop]{Proposição acerca de cones sobre esferas}, concluímos que $h$ é homeomorfismo e então\[S(S^n) \cong (H^{n+1} \amalg H^{n+1})/\sim \hspace{6 pt} \cong S^{n+1}.\]
    \end{dem}
\end{prop}

%\begin{titlemize}{Lista de consequências}
	%\item \hyperref[consequencia1]{Consequência 1}.
	%\item \hyperref[]{}
%\end{titlemize}


%%% Local Variables:
%%% mode: LaTeX
%%% TeX-master: "../Alg.Top-Wiki"
%%% End:

%---------------------------------------------------------------------------------------------------------------------!Draft!-----------------------------------------------------------------------------------------------------------------
\subsection{Espaços Quociente e a propriedade Hausdorff} %afirmação aqui significa teorema/proposição/colorário/lema
\label{topologia-quociente-hausdorff-thm}
\begin{titlemize}{Lista de dependências}
	\item \hyperref[topologia-quociente-def]{Espaços Quociente};\\ %'dependencia1' é o label onde o conceito Dependência 1 aparece (--à arrumar um padrão para referencias e labels--) 
% quantas dependências forem necessárias.
\end{titlemize}
Comentário sobre os objetos envolvidos na afirmação.
\begin{thm}[Espaços quocientes Hausdorff]% ou af(afirmação)/prop(proposição)/corol(corolário)/lemma(lema)/outros ambientes devem ser definidos no preambulo de Alg.Top-Wiki.tex 
Sejam $X$ um espaço Hausdorff e $\sim$ uma relação de equivalência em $X$ para a qual a projeção $\pi: X \rightarrow X/\sim$ é uma aplicação aberta. Defina o conjunto $R=\{(x,x')\in X\times X| x\sim x'\}$.

Então $X/\sim$ é Hausdorff se, e somente se, $R\subset X\times X$ é fechado.

\end{thm}
\begin{dem}
    $(\Longrightarrow)$ Se $X/\sim$ é de Hausdorff, gostaríamos de mostrar que $X\times X\backslash R$ é aberto. Para qualquer ponto $(x,x')\in (X\times X)\backslash R$, $x$ e $x'$ são tais que $\pi(x)\neq \pi(x')$. Como $X/\sim$ é de Hausdorff, existem abertos $U_x$ e $U_{x'}$ disjuntos em $X/\sim$ que são vizinhanças abertas de $\pi(x)$ e de $\pi(x')$, respectivamente.% e tais que $U_x\cap U_x' = \emptyset$.

    Temos ainda que $\pi^{-1}(U_x)$ e $\pi^{-1}(U_{x'})$ são abertos, pois a topologia de $X/\sim$ é a topologia quociente, e o produto $U=\pi^{-1}(U_x)\times \pi^{-1}(U_{x'})$ é aberto de $X\times X$ na topologia produto. Além disso, $(x,x')\in U$. Afirmamos que $U\subset X\times X\backslash R$. De fato, se $U\cap R\neq \emptyset$, teríamos $(v_1,v_2)\in U\cap R$ tal que $\pi(v_1)=\pi(v_2)$, mas $v_1 \in \pi^{-1}(U_x)$ e $v_2\in \pi^{-1}(U_{x'})$, e desse modo $\pi(v_1) = \pi(v_2) \in U_x \cap U_{x'} = \varnothing$, absurdo. Portanto, para todo $(x,x')\in X\times X\backslash R$, é possível encontrar uma vizinhança aberta $U$ de $(x,x')$ contida em $X\times X\backslash R$; $R$ é fechado, como queríamos.\newline

    $(\Longleftarrow)$ Dado que $R$ é fechado, gostaríamos de encontrar vizinhanças disjuntas de $a,~b\in X/\sim$ quaisquer para concluir que $X/\sim$ é Hausdorff. Sabemos que existem $x,~y\in X$ tais que $\pi(x)=a$ e $\pi(y)=b$ pois a projeção $\pi$ é uma aplicação sobrejetora. Como $X$ é de Hausdorff, existem abertos disjuntos $U_x$ e $U_y$, vizinhanças de $x$ e de $y$, respectivamente. Além disso, uma vez que $R$ é fechado, $X\times X\backslash R$ é aberto e, portanto, $(U_x\times U_y)\cap((X\times X)\backslash R)$ é aberto na topologia produto.

    Sejam $p_1:X\times X\rightarrow X$ e $p_2:X\times X\rightarrow X$ definidos por $$p_1(x_1,x_2)=x_1,\qquad p_2(x_1,x_2)=x_2 \qquad\forall (x_1,x_2)\in X\times X.$$ Como a topologia produto em $X\times Y$ é gerada pela base dada pelos produtos de abertos $X$ e de $Y$, é possível concluir que $\p_1$ e $\p_2$ são aplicações abertas. Desse modo, $U_1=p_1((U_x\times U_y)\cap((X\times X)\backslash R))$ e $U_2=p_2((U_x\times U_y)\cap((X\times X)\backslash R))$ são abertos em $X$. Por fim, basta observar que os abertos $\pi(U_1)$ e $\pi(U_2)$ são tais que $\pi(U_1)\cap \pi(U_2)=\emptyset$ uma vez que se $v\in \pi(U_1)\cap\pi(U_2)$, teríamos $v=\pi(v_1)$ para algum $v_1\in U_1$ e $v=\pi(v_2)$ para algum $v_2\in U_2$, o que implicaria $v_1\sim v_2$, um absurdo pois, pela construção de $U_1$ e $U_2$, $(v_1,v_2)\not\in R$.  Também temos $a\in U_1$ e $b\in U_2$ pois, como $a\neq b$, $x\not\sim y$. Encontramos assim os dois abertos que separam $a$ e $b$, mostrando que $X/\sim$ é Hausdorff.
\end{dem}

Comentários sobre a afirmação.

\begin{titlemize}{Lista de consequências}
	\item \hyperref[consequencia1]{Consequência 1};\\ %'consequencia1' é o label onde o conceito Consequência 1 aparece
\end{titlemize}


O espaço quociente também é essencial para realizar a colagem dos espaços.
\subsection{Pushout de espaços topológicos} %afirmação aqui significa teorema/proposição/colorário/lema
\label{pushout-de-espacos-topologicos-def}
\begin{titlemize}{Lista de dependências}
	\item \hyperref[topologia-quociente-def]{Espaços Quociente};\\ %'dependencia1' é o label onde o conceito Dependência 1 aparece (--à arrumar um padrão para referencias e labels--) 
% quantas dependências forem necessárias.
\end{titlemize}

\begin{defi}
    Sejam $X,Y,Z$ espaços topológicos. E Sejam $f:Z\rightarrow X$ e $g:Z\rightarrow Y$ funções contínuas. O \textbf{Pushout} de $f$ e $g$ é o espaço quociente $X\sqcup Y/\sim$, onde $\sim$ é a menor relação de equivalência que contém $\{(f(z),g(z))\in X\times Y:z\in Z\}$.
\end{defi}

%\begin{titlemize}{Lista de consequências}
	%\item %\hyperref[consequencia1]{Consequência 1};\\ %'consequencia1' é o label onde o conceito Consequência 1 aparece
%\end{titlemize}

\subsection{Colagem de n-célula} %afirmação aqui significa teorema/proposição/colorário/lema
\label{colagem-de-n-celula-def}
\begin{titlemize}{Lista de dependências}
	\item \hyperref[topologia-quociente-def]{Espaços Quociente};\\
    \item \hyperref[pushout-de-espacos-topologicos-def]{Pushout de espaços topológicos}.%'dependencia1' é o label onde o conceito Dependência 1 aparece (--à arrumar um padrão para referencias e labels--) 
% quantas dependências forem necessárias.
\end{titlemize}

\begin{defi}
    Seja $X$ um espaço topológico. E Sejam $f:\mathbb{S}^{n-1}\rightarrow X$ uma função contínua e $i:\mathbb{S}^{n-1}\hookrightarrow D^n$ uma inclusão, onde $n\ge 2$. O \textbf{espaço obtido de} $X$ \textbf{pela colagem de uma n-célula por meio da função } $f$ é pushout de $f$ e $i$, denotado por $X_f$ ou $D^n\cup_f X$.
\end{defi}

%\begin{titlemize}{Lista de consequências}
	%\item %\hyperref[consequencia1]{Consequência 1};\\ %'consequencia1' é o label onde o conceito Consequência 1 aparece
%\end{titlemize}

\subsection{Colagem de um disco com um ponto} %afirmação aqui significa teorema/proposição/colorário/lema
\label{colagem-de-um-disco-com-um-ponto-ex}
\begin{titlemize}{Lista de dependências}
	\item \hyperref[topologia-quociente-def]{Espaços Quociente};\\
    \item \hyperref[pushout-de-espacos-topologicos-def]{Pushout de espaços topológicos};\\
    \item \hyperref[colagem-de-n-celula-def]{Colagem de n-célula}%'dependencia1' é o label onde o conceito Dependência 1 aparece (--à arrumar um padrão para referencias e labels--) 
% quantas dependências forem necessárias.
\end{titlemize}

\begin{ex}
    Dado $n\ge 2$, sejam $N = (0,\ldots,0,1) \in \mathbb{S}^n$ e $S = -N$. A colagem $\{x\}_f=D^n\cup_f \{x\}$, em que $f:\mathbb{S}^{n-1}\rightarrow \{x\}$ é a função constante, é homeomorfa à esfera $\mathbb{S}^n$. 
\end{ex}

\begin{dem}
    Note que $\text{int}(D^n)\cong \mathbb{R}^n\cong \mathbb{S}^n\setminus\{N\}$. Seja $g_0:\text{int}(D^n)\rightarrow \mathbb{R}^n$ um homeomorfismo. Podemos estender $g_0$ na seguinte forma 
    \begin{align*}
        g:\{x\}_f&\longrightarrow \mathbb{S}^n\\
        p&\longmapsto g(p)=\begin{cases}
            g_0(p) &\text{ se }p\ne [x]\\
            N &\text{ se }p=[x].
        \end{cases}
    \end{align*}
    Essa função é bem-definida e bijetora. Agora, para mostrar que $g$ é um homeomorfismo, basta mostrar que $g$ é contínua e aberta.\\
    A função $g$ é contínua: A função $g$ é contínua se, e somente se, $g\circ \pi$ é contínua, onde $\pi:D^n\bigsqcup \{x\}\rightarrow \{x\}_f$ é a projeção associada ao quociente. A função $g\circ \pi$ é contínua, pois dado um aberto $U$ de $\mathbb{S}^n$, temos:
    \begin{itemize}
        \item se $N\notin U$, então $(g\circ\pi)^{-1}(U)=g_0^{-1}(U)$, que é aberto;
        \item se $N\in U$, então $(g\circ \pi)^{-1}(U)=g_0^{-1}(U\setminus\{N\})\cup \{x\}$, que é um aberto, pois $x$ é um ponto isolado.
    \end{itemize}
    Portanto, $g$ é contínua.\\
    A função $g$ é aberta: Seja $U$ um aberto de $\{x\}_f$. Teremos dois casos: 
    \begin{itemize}
        \item Se $x\notin U$, então $g(U)=g_0(U)$, que é um aberto em  $\mathbb{S}^n\setminus\{N\}$. Assim, ou $g_0(U)$ é aberto em $\mathbb{S}^n$, ou $g_0(U)\cup\{N\}$ é aberto em $\mathbb{S}^n$. Se $g_0(U)$ for aberto, então $g(U)$ será um aberto em $\mathbb{S}^n$. Se $g_0(U)\cup \{N\}$ for aberto, então $g_0(U)=(g_0(U)\cup\{N\})\setminus\{N\}$ é um aberto em $\mathbb{S}^n$. Em ambos os casos, $g(U)$ será um aberto em $\mathbb{S}^n$;
        \item Se $x\in U$, então $\pi^{-1} (U)$ é um aberto em $D^n\sqcup \{x\}$. Pela construção do quociente, temos $\partial D^n\subseteq\pi^{-1}(U)$, logo, para todo ponto $y\in \partial D^n$, existe um $r_y>0$ tal que 
        $$B_y:=\{z\in D^n: ||z-y||<r_y\}\subseteq \pi_1^{-1}(U).$$
        A coleção $\{B_y\}_{y\in \partial D^n}$ é uma cobertura aberta de $\partial D^n$. Como o bordo $\partial D^n$ é compacto, existem $y_1,...,y_k$ tal que 
        \[\partial D^n\subseteq B_{y_1}\cup...\cup B_{y_k}.\]
        Considere $r=\text{inf}\{r_{y_1},...,r_{y_k}\}$. Assim, o conjunto 
        \[B=\pi(\{y\in D^n: ||y||>(1-r)\}\cup\{x\})\subseteq U\]
        é um aberto em $\{x\}_f$, e $g(B)\subseteq g(U)$ corresponde a uma bola centrada em $N$ em $\mathbb{S}^n$. Note que $U\setminus \{x\}$ é um aberto, pois $x$ é um ponto fechado. Pelo item anterior, temos que o aberto
        \[g(U)=g(B)\cup g(U\setminus\{x\})\]
        é uma união de abertos, o que implica que $g(U)$ é um aberto em $\mathbb{S}^n$
    \end{itemize} 
    Portanto, $g$ é aberta.
\end{dem}

%%% Local Variables:
%%% mode: LaTeX
%%% TeX-master: "../Alg.Top-Wiki"
%%% End:

