\subsection{Colagem de um disco com um ponto} %afirmação aqui significa teorema/proposição/colorário/lema
\label{colagem-de-um-disco-com-um-ponto-ex}
\begin{titlemize}{Lista de dependências}
	\item \hyperref[topologia-quociente-def]{Espaços Quociente};\\
    \item \hyperref[pushout-de-espacos-topologicos-def]{Pushout de espaços topológicos};\\
    \item \hyperref[colagem-de-n-celula-def]{Colagem de n-célula}%'dependencia1' é o label onde o conceito Dependência 1 aparece (--à arrumar um padrão para referencias e labels--) 
% quantas dependências forem necessárias.
\end{titlemize}

\begin{ex}
    Para $n\ge 2$, a colagem $\{x\}_f=D^n\cup_f \{x\}$, em que $f:\mathbb{S}^{n-1}\rightarrow \{x\}$ é a função constante, é a esfera $\mathbb{S}^n$. Visto que $int(D^n)$ é homeomorfo a $\mathbb{R}^n$, e $\mathbb{R}^n$ é homeomorfo a $\mathbb{S}^n\setminus\{\text{polo norte}\}$, e que a colagem garante que $\{x\}_f\setminus\{\text{origem do disco }D^n \}$ é homeomorfo a $\mathbb{S}^n\setminus\{\text{polo sul}\}$, conclui-se que $\{x\}_f$ é homeomorfo à esfera $\mathbb{S}^n$.
\end{ex}

%\begin{titlemize}{Lista de consequências}
	%\item %\hyperref[consequencia1]{Consequência 1};\\ %'consequencia1' é o label onde o conceito Consequência 1 aparece
%\end{titlemize}
