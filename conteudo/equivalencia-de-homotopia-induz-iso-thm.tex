\subsection{Equivalência de homotopia e o grupo fundamental}
\label{equiv-homotopia-induz-iso}
\begin{titlemize}{Lista de dependências}
    \item \hyperref[equiv-homotopia]{Equivalência de Homotopia};
	\item \hyperref[hom-grupo-fundamental]{Homomorfismo de grupos fundamentais};
    \item \hyperref[conjugacao-por-curva-prop]{Conjugação por uma Curva}
\end{titlemize}

\begin{lemma}
    Sejam $X$ um espaço topológico, $x_0\in X$ e $\phi: X \to X$ uma função contínua tal que $\phi\sim \text{id}_X$. Então $\phi_*: \pi_1(X,x_0)\to \pi_1(X,\phi(x_0))$ é isomorfismo de grupos.

    \begin{dem}
        Denotemos $x_1=\phi(x_0)$. Tomemos $H(x,t): X\times I \to X$ uma homotopia de $\text{id}_X$ à $\phi$. Então a curva $\gamma: I \to X$ dada por $\gamma(t) = H(x_0,t)$ liga $x_0$ a $x_1$.
        
        Afirmamos que $\phi_* = \hat{A}_{\gamma}$; ou seja, para toda $\eta \in \Omega(X,x_0)$ vale que $\phi(\eta) \sim \overline{\gamma} * \eta * \gamma$ relativa a $\partial I$. De fato, $H(\eta(.),.): I\times I \to X$ é contínua, $H(\eta(s),0)=\eta(s)$ e $H(\eta(s),1)=\phi(\eta)(s)$ para todo $s\in I$, sendo assim uma homotopia entre $\eta$ e $\phi(\eta)$. Definindo então\begin{align*}
            H': I\times I&\to X\\
                (s,t)&\mapsto \begin{cases}
                \overline{\gamma}(s),&\text{ se }s\in[0,\frac{t}{3}]\\
                H\left(\eta\left(\frac{3s-t}{3-2t}\right),t\right),&\text{ se }s\in[\frac{t}{3},1-\frac{t}{3}]\\
                \gamma(s),&\text{ se }s\in[1-\frac{t}{3},1]
            \end{cases}
        \end{align*}
        concluímos que $\phi(\eta) \underset{H'}{\sim} \overline{\gamma} * \eta * \gamma$ relativa a $\partial I$.
    \end{dem}
\end{lemma}


\begin{thm}
    Sejam $X$ e $Y$ espaços topológicos, $x_0\in X$ e $f:X\to Y$ uma equivalência de homotopia. Então, $f_*: \pi_1(X,x_0) \to \pi_1(X,f(x_0))$ é isomorfismo de grupos.
\end{thm}

Tal resultado não é tão imediato quanto se parece. Tomando $g: Y\to X$ inversa a menos de homotopia de $f$, $g \circ f \sim \text{id}_X$, porém disso não segue que $f_* \circ g_* = \text{id}_{\pi_1(X,x_0)}$, pois $g_* \circ f_*: \pi_1(X,x_0) \to \pi_1(X,g\circ f(x_0))$. E mesmo que $g\circ f(x_0) = x_0$, a homotopia $H: g\circ f\Rightarrow \text{id}_X$ não precisa ser relativa à $\{x_0\}$.

\begin{dem}
    Sejam $y_0 = f(x_0)$, $x_1 = g(y_0) = g\circ f(x_0)$ e $y_1 = f(x_1) = f\circ g\circ f(x_0)$. Sabemos que as composições
    \[\pi_1(X,x_0) \xrightarrow{f_{*,x_0}} \pi_1(Y,y_0) \xrightarrow{g_{*,y_0}}\pi_1(X,x_1)\]
    e
    \[\pi_1(Y,y_0) \xrightarrow{g_{*,y_0}} \pi_1(X,x_1) \xrightarrow{f_{*,x_1}}\pi_1(Y,y_1)\]
    são homotópicas a $\text{id}_X$ e $\text{id}_Y$, respectivamente. Com tal notação, explicitamos que $f_{*,x_0}$ e $f_{*,x_1}$ estão definidas sobre domínios diferentes. Pelo Lema anterior, $g_{*,y_0}\circ f_{*,x_0}$ e $f_{*,x_1}\circ g_{*,y_0}$ são isomorfismos de grupos. Em especial, $f_{*,x_0}$ é injetora e $f_{*,x_1}$ é sobrejetora.

    Porém a Nota na Seção Conjugação por uma Curva nos garante que $f_{*,x_0} = \hat{A}_{f(\gamma)}^{-1} \circ f_{*,x_1} \circ \hat{A}_{\gamma}$ é composição de funções sobrejetoras, portanto é sobrejetora também (o mesmo argumento garante que $f_{*,x_1}$ é injetora). Como já sabíamos que $f_{*,x_0}$ é homomorfismo de grupos, concluímos que $f_{*,x_0}:\pi_1(X,x_0)\to \pi_1(Y,y_0)$ é isomorfismo de grupos.
\end{dem}


%\begin{titlemize}{Lista de consequências}
	%\item \hyperref[consequencia1]{Consequência 1};\\ %'consequencia1' é o label onde o conceito Consequência 1 aparece
	%\item \hyperref[]{}
%\end{titlemize}
