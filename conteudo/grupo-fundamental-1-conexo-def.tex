%---------------------------------------------------------------------------------------------------------------------!Draft!-----------------------------------------------------------------------------------------------------------------
\subsection{Espaço 1-conexo}
\label{espaço-1-conexo-def}
\begin{titlemize}{Lista de dependências}
	\item \hyperref[grupo-fundamental-def]{Grupo fundamental};\\ %'dependencia1' é o label onde o conceito Dependência 1 aparece (--à arrumar um padrão para referencias e labels--) 
	%\item \hyperref[]{};\\
% quantas dependências forem necessárias.
\end{titlemize}

\begin{defi}[Espaço 1-conexo]
	$E$ é um espaço \textbf{1-conexo} se é conexo por caminhos e se $\pi_1(E, e_0)=\{1\}$.
\end{defi}

A definição indica que, além de o espaço ser conexo por caminhos, se forem dados quaisquer dois caminho entre $x_0$ e $y_0$ fixados, é possível levar um continuamente ao outro de forma que $x_0$ e $y_0$ permanecem fixados no processo, conforme indica a proposição \ref{1-conexo-prop}.

\begin{titlemize}{Lista de consequências}
	\item \hyperref[recobrimento-1-conexo-prop]{Teorema dos recobrimentos 1-conexos};\\ %'consequencia1' é o label onde o conceito Consequência 1 aparece
	\item \hyperref[recobrimento-1-conexo-em-bijecao-com-P(X,x)]{Recobrimento 1-conexo em bijeção com $P(X,x_0)$}
 	\item \hyperref[descrição-da-base-do-recobrimento-prop]{Descrição de $\tilde{\mathcal{B}}$ em termos de $X$}
\end{titlemize}
