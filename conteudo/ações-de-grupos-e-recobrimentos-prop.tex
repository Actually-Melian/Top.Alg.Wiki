%-------------------------------------------------------------------------------------------------------------!Draft!-------------------------------------------------------------------------------------------------------------------------
\subsection{Ações de Grupos e Recobrimentos}
\label{ações-de-grupos-e-recobrimentos-prop}

\begin{titlemize}{Lista de Dependências}
	\item \hyperref[ações-de-grupo-def]{Ações de Grupos}; %'dependencia1' é o label onde o conceito Dependência 1 aparece (--à arrumar um padrão para referencias e labels--) 
    \item \hyperref[ações-de-grupo-propriamente-descontínuas-def]{Ações de grupos propriamente descontínuas}% quantas dependências forem necessárias.
 
\end{titlemize}

Uma pergunta natural que surge ao longo do estudo de topologia algébrica é:
Para quais ações  G $\circlearrowright X$ temos que a aplicação quociente $p: X \rightarrow\mkern-18mu \quad{^{\textstyle X}\big/_{\textstyle G}}$ é um recobrimento?\\\\
onde $\quad{^{\textstyle X}\big/_{\textstyle G}} =\mkern-18mu \quad{^{\textstyle X}\big/_{\textstyle \sim}}$, definindo $\sim$ como a seguinte relação de equivalência: \[x\sim y \Leftrightarrow \exists \ g \in G \mid y = gx\]
\\
Assim, obtemos a seguinte proposição:\\

\begin{thm}[Ações propriamente descontínuas geram recobrimentos]% ou af(afirmação)/prop(proposição)/corol(corolário)/lemma(lema)/outros ambientes devem ser definidos no preambulo de Alg.Top-Wiki.tex 
	Seja \\ $\Psi:G \times X \longrightarrow X$ uma ação propriamente descontínua. Então a aplicação quociente
 $p: X \longrightarrow\mkern-18mu \quad{^{\textstyle X}\big/_{\textstyle G}}$ é um recobrimento.
\end{thm}

\begin{dem}
    Vamos mostrar que dado $[x] \in\mkern-18mu \quad{^{\textstyle X}\big/_{\textstyle G}}$, existe $U \subset\mkern-18mu \quad{^{\textstyle X}\big/_{\textstyle G}}$ \\
    vizinhança uniformemente recoberta de $[x]$:\\\\
    Note que \[p^{-1}(U) = \coprod_{g \in G} V_g \] onde $V_g \subset X $ são abertos e ainda:
    \[p\restriction_{V_g}:V_g \rightarrow U\] é uma relação de equivalência.\\
    Dado $x \in X$, seja $V_1$ vizinhança de $x$ tal que $gV_1\cap V_1 \neq \emptyset \Rightarrow g = 1$\\
    Sejam agora $V_g = gV_1$ e $U = p(V_1)$, vizinhança de $[x]$\\
    Então temos:\\
    $$
        \begin{cases}
            p^{-1}(U) = \coprod_{g\in G} V_g\\
            p\restriction_{V_1}:V_1 \rightarrow U \text{ é homeomorfismo}
        \end{cases}
    $$\\
    Falta mostrar que $p\restriction_{V_g}$ é um homeomorfismo. Mas temos que:

    \[
        p\restriction_{V_g} = p\restriction_{V_1} \circ \Psi_{g^{-1}}
    \]
    É composição de homeomorfismos
\end{dem}

\begin{titlemize}{Lista de consequências}
	%'consequencia1' é o label onde o conceito Consequência 1 aparece
    \item \hyperref[ações-de-grupos-e-gr-fundamental-prop]{Encontrando o grupo fundamental através de uma ação propriamente descontínua}
\end{titlemize}
%[Bianca]: Decidir o que é um assunto e o que é apenas um resultado/definição pode ser uma tarefa não trivial. A princípio essa tarefa será realizada pelos alunos, em conjunto. Se for necessário podem entrar em contato com o professor Ivan ou comigo. 


% Cada novo assunto deve ser adcionado no corpo do texto, como explicado no arquivo Alg.Top-Wiki.tex.
