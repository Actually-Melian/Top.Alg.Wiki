%---------------------------------------------------------------------------------------------------------------------!Draft!-----------------------------------------------------------------------------------------------------------------
\subsection{Suspensão coincide com Cone Duplo}
\label{suspensao-cone-duplo-prop}
\begin{titlemize}{Lista de dependências}
	\item \hyperref[cone-def]{Cone};\\
	\item \hyperref[suspensao-def]{Suspensão}.
\end{titlemize}
Provaremos que a suspensão sobre um espaço topológico é homeomorfa a um cone duplo, confirmando as intuições sobre os objetos.
\begin{prop}[A construção de suspensão coincide com a de cone duplo]
	Dado $X$ um espaço topológico, considere os espaços
    \[S(X) = X\times[-1,1]/\sim_S\hspace{6 pt} = \{[x,t]_S : x\in X, t\in [-1,1]\},\]
    \[C(X) = X\times I/\sim_C\hspace{6 pt} = \{[x,t]_C : x\in X, t\in I\}.\]
    \\
    \noindent E sobre $C(X) \amalg C(X) = C(X)\times\{-1,1\}$, seja $\sim$ a relação de equivalência\\
    \\ \centerline{
    $([x,s]_C,i) \sim ([y,t]_C,j) \Leftrightarrow ([x,s]_C,i) = ([y,t]_C,j)$ ou $(x=y$ e $s=t=0)$}\\
    \\para todos $x,y \in X$, $s,t \in I$ e $i,j \in \{-1,1\}$.    
    
    Defina a função \begin{align*}
        \psi: S(X) &\rightarrow (C(X) \amalg C(X))/ \sim\\
                [x,t]_S &\mapsto [[x,|t|]_C,\text{sgn}(t)]
    \end{align*}
    onde $\text{sgn}(t) = -1$ caso $t < 0$ e $\text{sgn}(t) = 1$ caso contrário. Então $\psi$ é homeomorfismo.

	Ou seja, a construção de suspensão sobre um espaço topológico coincide com a colagem de dois cones sobre o mesmo espaço topológico, quando identificamos as suas respectivas bases.

    \begin{dem}
        Primeiramente, $\psi$ está bem definida. Dado $(x,t) \in X\times I$, vale que $[x,t]_S = \{(x,t)\}$, se $t \in \left]-1,1\right[$, ou então $[(x,t)] = X\times\{t\}$, se $t \in \{-1,1\}$. Mas para quaisquer $x,x'\in X$ e $t\in \{-1,1\}$,\[
        [[x,|t|]_C,\text{sgn}(t)] =
        [X\times\{|t|\},\text{sgn}(t)] =
        [[x',|t|]_C,\text{sgn}(t)]
        \]
        e então está bem definido $\psi([x,t]_S)$.

        Denotemos por $\pi_S$, $\pi_C$ e $\pi$ as aplicações quociente canônicas com respeito a $\sim_S$, $\sim_C$ e $\sim$, respectivamente. Por \textbf{(adicionar proposição aqui)}, $\psi$ é contínua se, e somente se,\begin{align*}
            \psi \circ \pi_S: X\times[-1,1] &\rightarrow (C(X) \amalg C(X))/ \sim\\
            (x,t) &\mapsto [[x,|t|]_C,\text{sgn}(t)]
        \end{align*}
        for contínua. Para todo subconjunto aberto $U \subset C(X)\amalg C(X)/\sim\hspace{3pt}$, existem abertos $V,W \subset C(X)$ tais que $\pi^{-1}(U) = V\amalg W = (V\times\{-1\}) \cup (W\times\{1\})$. Por sua vez, $\pi_C^{-1}(V)$ e $\pi_C^{-1}(W)$ são abertos de $X\times[0,1]$. Então,
        \begin{align*}
        (\psi \circ \pi_S)^{-1}(U)
        &= (\psi \circ \pi_S)^{-1}(\pi(V\times\{-1\}) \cup \pi(W\times\{1\}))\\
        &= (\psi \circ \pi_S)^{-1}(\pi(V\times\{-1\})) \cup (\psi \circ \pi_S)^{-1}(\pi(W\times\{1\})).   
        \end{align*}
        
        Calculemos cada termo da união separadamente.
        \begin{align*}
            (\psi \circ \pi_S)^{-1}(\pi(V\times\{-1\}))
            &= \{(x,t): [x,|t|]_C \in V, t \in [-1,0]\}\\
            &= \{(x,-t): [x,t]_C \in V, t \in I\}\\
            &= A(\pi_C^{-1}(V))   
        \end{align*}
        onde $A: (x,t) \mapsto (x,-t)$. E analogamente, $(\psi \circ \pi_S)^{-1}(\pi(W\times\{1\})) = \pi_C^{-1}(W)$. Concluímos assim que
        \[(\psi \circ \pi_S)^{-1}(U) = A(\pi_C^{-1}(V)) \cup \pi_C^{-1}(W)\]
        é aberto, e então $\psi$ é contínua.
    \end{dem}
\end{prop}


\begin{titlemize}{Lista de consequências}
	%\item \hyperref[consequencia1]{Consequência 1};\\ %'consequencia1' é o label onde o conceito Consequência 1 aparece
	%\item \hyperref[]{}
    \item \hyperref[suspensao-euclidiano-prop]{Suspensão sobre subespaços de $\mathbb{R}^n$}
    \item \hyperref[suspensao-esfera-prop]{Suspensão sobre esferas $S^n\subset\mathbb{R}^{n+1}$}
\end{titlemize}

