%---------------------------------------------------------------------------------------------------------------------!Draft!-----------------------------------------------------------------------------------------------------------------
\subsection{Espaço semi-localmente simplesmente conexo}
\label{espaço-semi-localmente-simplesmente-conexo-def}
\begin{titlemize}{Lista de dependências}
	\item \hyperref[grupo-fundamental-def]{Grupo fundamental};\\ %'dependencia1' é o label onde o conceito Dependência 1 aparece (--à arrumar um padrão para referencias e labels--) 
	\item \hyperref[hom-grupo-fundamental]{Homomorfismo de grupos fundamentais};\\
% quantas dependências forem necessárias.
\end{titlemize}
\begin{defi}[Espaço semi-localmente simplesmente conexo]
	$X$ é um espaço semi-localmente simplesmente conexo se para todo $x\in X$ existe $U$ vizinhança de $x$ tal que $$i_*: \pi_1(U,x)\rightarrow\pi_1(X,x)$$ é trivial, isto é, para quaisquer duas classes $[\alpha],[\beta] \in \pi_1(U,x)$ temos que $[i\circ \alpha]=[i\circ \beta]\in \pi_1(X,x)$.
\end{defi}
 

\begin{titlemize}{Lista de consequências}
    \item \hyperref[não-semi-localmente-simplesmente-conexo-ex]{Exemplo de espaço não semi-localmente simplesmente conexo}
	\item \hyperref[recobrimento-1-conexo-prop]{Teorema dos recobrimentos 1-conexos};\\ %'consequencia1' é o label onde o conceito Consequência 1 aparece
\end{titlemize}

%[Bianca]: é mais fácil criar a lista de dependências do que a de consequências.