%---------------------------------------------------------------------------------------------------------------------!Draft!-----------------------------------------------------------------------------------------------------------------
\subsection{Transformação Natural}
\label{transformação-natural-categorias-def}
\begin{titlemize}{Lista de dependências}
	\item \hyperref[funtor-categorias-def]{funtor-categorias-def};\\ %'dependencia1' é o label onde o conceito Dependência 1 aparece (--à arrumar um padrão para referencias e labels--) 
	\item \hyperref[]{};\\
% quantas dependências forem necessárias.
\end{titlemize}
\begin{defi}[Transformação Natural]
	Dados dois funtores % https://q.uiver.app/#q=WzAsMixbMCwwLCJcXG1hdGhjYWx7Q30iXSxbMSwwLCJcXG1hdGhjYWx7RH0iXSxbMCwxLCJGIiwwLHsib2Zmc2V0IjotMX1dLFswLDEsIkciLDIseyJvZmZzZXQiOjF9XV0=
\begin{tikzcd}[cramped,sep=small]
	{\mathcal{C}} & {\mathcal{D}}
	\arrow["F", shift left, from=1-1, to=1-2]
	\arrow["G"', shift right, from=1-1, to=1-2]
\end{tikzcd}, definimos a transformação natural $\eta:F \Longrightarrow G$ da seguinte forma: \\
$\eta$ é uma família de flechas $(\eta_A: F(A) \longrightarrow G(A))_{A \in Obj(\mathcal{C})}$, tal que $\eta_B \circ F(f) = G(f) \circ \eta_A$. Isso equivale a dizer que o seguinte diagrama comuta.
% https://q.uiver.app/#q=WzAsNixbMiwwLCJGKEEpIl0sWzQsMCwiRyhBKSJdLFsyLDIsIkYoQikiXSxbNCwyLCJHKEIpIl0sWzAsMCwiQSJdLFswLDIsIkIiXSxbNCw1LCJmIiwyXSxbMCwyLCJGKGYpIiwyXSxbMSwzLCJHKGYpIiwyXSxbMCwxLCJcXGV0YV9BIiwxXSxbMiwzLCJcXGV0YV9CIiwxXV0=
\[\begin{tikzcd}[sep=large]
	A && {F(A)} && {G(A)} \\
	\\
	B && {F(B)} && {G(B)}
	\arrow["f"', from=1-1, to=3-1]
	\arrow["{\eta_A}"{description}, from=1-3, to=1-5]
	\arrow["{F(f)}"', from=1-3, to=3-3]
	\arrow["{G(f)}"', from=1-5, to=3-5]
	\arrow["{\eta_B}"{description}, from=3-3, to=3-5]
\end{tikzcd}\]

    
\end{defi}

A transformação natural identidade é a transformação $1_F:F \Longrightarrow F$, tal que $(1_F)_A: F(A) \longrightarrow F(A)$ é a identidade de $F(A)$. \\
Ainda, dadas as trasformações naturais $\eta:F \Longrightarrow G$ e $\mu: G \Longrightarrow H$, onde $F, G$ e $H$ são funtores de uma categoria $\mathcal{C}$ para uma categoria $\mathcal{D}$, podemos definir a transformação $(\mu \circ \eta): F \Longrightarrow H$ como sendo a família de flechas $(\mu_A \circ \eta_A: F(A) \longrightarrow G(A))_{A \in Obj(\mathcal{C})}$.

Dessa forma, podemos definir o que é a categoria de funtores: 

$\mathbf{Fun(\mathcal{C}, \mathcal{D})}$ é a categoria em que os objetos são funtores de $\mathcal{C}$ para $\mathcal{D}$ e os morfismos são transformações naturais dos funtores de $Obj(\mathbf{Fun(\mathcal{C}, \mathcal{D})})$.

\begin{titlemize}{Lista de consequências}
	\item \hyperref[grupo-fundamental]{grupo-fundamental};\\ %'consequencia1' é o label onde o conceito Consequência 1 aparece
	\item \hyperref[homotopia]{homotopia}
\end{titlemize}

%[Bianca]: é mais fácil criar a lista de dependências do que a de consequências.
