%---------------------------------------------------------------------------------------------------------------------!Draft!-----------------------------------------------------------------------------------------------------------------
\subsection{Suspensão sobre subespaços de $\mathbb{R}^n$}
\label{suspensao-euclidiano-prop}
\begin{titlemize}{Lista de dependências}
	\item \hyperref[suspensao-def]{Suspensão sobre um Espaço Topológico};
\end{titlemize}
Provaremos que a construção de uma suspensão sobre um subespaço topológico de $\mathbb{R}^n$ (com a topologia usual) é homeomorfa à construção geométrica em $\mathbb{R}^{n+1}$, sendo assim uma generalização de conceitos que antes só faziam sentido no contexto de espaços euclidianos.
\begin{thm}[A construção de suspensão generaliza a de subespaços euclidianos]
	Seja $X \subset \mathbb{R}^n$. Então,\begin{align*}
        \varphi: S(X) &\rightarrow \{((1-|t|) x_1,...,(1-|t|) x_n,t): (x_1,...,x_n) \in X, t \in [-1,1]\}\subset \mathbb{R}^{n+1}\\
        [(x,t)] \mapsto ((1-t) x_1,...,(1-t) x_n,t), \forall x=(x_1,...,x_n)\in X, \forall t \in [-1,1]
    \end{align*}
    é um homeomorfismo.\\
    Ou seja, para subespaços de $\mathbb{R}^n$, a construção de cone recupera a intuição geométrica e coincide enquanto espaço topológico com $\{((1-|t|)x,t):x\in X, t\in [-1,1]\} \in \mathbb{R}^{n+1}$, onde $(x,t)$ denota concatenação de vetores.
\end{thm}

Tal proposição é análoga à sobre \hyperref[cone-euclidiano-prop]{cone sobre um espaço euclidiano}.

\begin{titlemize}{Lista de consequências}
	\item \hyperref[consequencia1]{Consequência 1};\\ %'consequencia1' é o label onde o conceito Consequência 1 aparece
	\item \hyperref[]{}
\end{titlemize}

%[Bianca]: Um arquivo tex pode ter mais de uma afirmação (ou definição, ou exemplo), mas nesse caso cada afirmação deve ter seu próprio label. Dar preferência para agrupar afirmações que dependam entre sí de maneira próxima (um teorema e seu corolário, por exemplo)
